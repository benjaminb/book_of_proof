\documentclass{article}
\usepackage{amsmath} % for advanced math environments
\usepackage{amsfonts} % for math fonts
\usepackage{amssymb} % for math symbols
\usepackage{amsthm} % for theorems and proofs
\usepackage{mathtools} % for mathematical tools
\usepackage{mathrsfs} % for script-like fonts in math
\usepackage{bm} % for bold math symbols
\usepackage{bbm} % for "blackboard-style" characters in math
\usepackage{graphicx} % for including graphics
\usepackage{hyperref} % for including hyperlinks
\usepackage{tcolorbox}
\usepackage{tikz}
\tcbuselibrary{theorems, breakable}
\usepackage{xcolor}
\usepackage[margin=1in]{geometry}

\newcommand{\C}{\mathbb{C}}
\newcommand{\N}{\mathbb{N}}
\newcommand{\Q}{\mathbb{Q}}
\newcommand{\R}{\mathbb{R}}
\newcommand{\Z}{\mathbb{Z}}
\newcommand{\pset}{\mathscr{P}}
\DeclareMathOperator{\lcm}{lcm}

% Define a shortcut for \begin{bmatrix} and \end{bmatrix}
\newcommand{\bmat}[1]{\begin{bmatrix}#1\end{bmatrix}}
\newcommand{\cmat}[1]{\begin{pmatrix}#1\end{pmatrix}}

\newtcolorbox[auto counter]{problem}%
{
    breakable,
    colback=cyan!5,
    colframe=cyan!35!black,
    fonttitle=\bfseries,
    title=Problem~\thetcbcounter,
}

\newtcolorbox{solution}[1]
{
    breakable,
    colback=red!5,
    colframe=red!75!black,
    fonttitle=\bfseries,
    title=Solution: #1,
}

% Title
\title{Your Document Title}
\author{Your Name}
\date{\today} % or specify a date like {December 2020}

\begin{document}

\maketitle
\begin{problem}
Prove that the function $f(x) = \sqrt{x}$ is continuous at any number $c > 0$. Deduce that $\lim\limits_{x\to c}\sqrt{g(x)} = \sqrt{\lim\limits_{x \to c} g(x)}$, provided $\lim\limits_{x \to c} g(x)$ exists and is greater than 0.
\end{problem}

First we show $\sqrt{x}$ is continuous using and $\epsilon-\delta$ proof. For scratchwork:
\begin{align*}
  |\sqrt{x} - \sqrt{c} |                                                       & < \epsilon                      \\
  |(\sqrt{x} - \sqrt{c})\cdot \frac{\sqrt{x} + \sqrt{c}}{\sqrt{x} + \sqrt{c}}| & < \epsilon                      \\
  |\frac{x - c}{\sqrt{x} + \sqrt{c}}|                                          & < \epsilon                      \\
  |x - c|                                                                      & < \epsilon(\sqrt{x} + \sqrt{c}) \\
  |x - c|                                                                      & < \epsilon\sqrt{c}              \\
\end{align*}

At the last line, we can drop the $\sqrt{x}$ term multiplying $\epsilon$ because we can make $x$ as close to $c$ as we like, and since the domain is positive numbers the roots are positive. Therefore $\epsilon\sqrt{c} < \epsilon(\sqrt{x} + \sqrt{c})$ for any $x > 0$. Now write the formal proof: for any $\epsilon > 0$ set $\delta < \epsilon\sqrt{c}$. Then for $0 < |x - c| < \delta$ we have
\begin{align*}
  |f(x) - f(c)| & = |\sqrt{x} - \sqrt{c}|               \\
                & = \frac{|x - c|}{\sqrt{x} + \sqrt{c}} \\
                & < \frac{\delta}{\sqrt{c}}             \\
                & < \frac{\epsilon\sqrt{c}}{\sqrt{c}}   \\
                & = \epsilon
\end{align*}

Now that we know the square root function is continuous at any positive real number, we can apply the theorem that
$$\lim_{x \to c} \sqrt{g(x)} = \sqrt{\lim_{x \to c} g(x)}$$

provided the root function is continuous at $g(c)$.

\begin{problem}
Show that the condition of continuity in Theorem 13.9 is necessary by finding functions $f$ and W$g$ for which $\lim\limits_{x\to c}g(x) = L$, and $f$ is \textit{not} continuous at $x = L$, and $\lim\limits_{x \to c} f(g(x)) \neq f\left(\lim\limits_{x \to c} g(x)\right)$.
\end{problem}

Theorem 13.9 is that continuous functions compose:
$$\lim_{x \to c} f(g(x)) = f\left(\lim_{x \to c} g(x)\right)$$

The problem statement asks to show a function that is not continuous at $L$. One simple solution is to define a piecewise function with a point discontinuity:
$$f(x) = \begin{cases} x & x \neq 0 \\ \pi & x = 0 \end{cases}$$

This $f$ is almost the identity function except at 0, it jumps to $\pi$. Combine that with $g(x) = x$, the actual identity function. Then
$$\lim_{x \to 0} f(g(x)) = \lim_{x \to 0} f(x) = 0$$

Since as $x$ approaches 0 (but does not equal 0), $f(x)$ using the `identity' branch, approaches 0 as well. However if we take the limit of $g$ before sending to $f$ we get $f(0) = \pi$.

\end{document}