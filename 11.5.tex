\documentclass{article}
\usepackage{amsmath} % for advanced math environments
\usepackage{amsfonts} % for math fonts
\usepackage{amssymb} % for math symbols
\usepackage{amsthm} % for theorems and proofs
\usepackage{mathtools} % for mathematical tools
\usepackage{mathrsfs} % for script-like fonts in math
\usepackage{bm} % for bold math symbols
\usepackage{bbm} % for "blackboard-style" characters in math
\usepackage{graphicx} % for including graphics
\usepackage{hyperref} % for including hyperlinks
\usepackage{tcolorbox}
\usepackage{tikz}
\tcbuselibrary{theorems, breakable}
\usepackage{xcolor}
\usepackage[margin=1in]{geometry}

\newcommand{\C}{\mathbb{C}}
\newcommand{\N}{\mathbb{N}}
\newcommand{\Q}{\mathbb{Q}}
\newcommand{\R}{\mathbb{R}}
\newcommand{\Z}{\mathbb{Z}}
\newcommand{\pset}{\mathscr{P}}
\DeclareMathOperator{\lcm}{lcm}

% Define a shortcut for \begin{bmatrix} and \end{bmatrix}
\newcommand{\bmat}[1]{\begin{bmatrix}#1\end{bmatrix}}
\newcommand{\cmat}[1]{\begin{pmatrix}#1\end{pmatrix}}

\newtcolorbox[auto counter]{problem}%
{
    breakable,
    colback=cyan!5,
    colframe=cyan!35!black,
    fonttitle=\bfseries,
    title=Problem~\thetcbcounter,
}

\newtcolorbox{solution}[1]
{
    breakable,
    colback=red!5,
    colframe=red!75!black,
    fonttitle=\bfseries,
    title=Solution: #1,
}

% Title
\title{Your Document Title}
\author{Your Name}
\date{\today} % or specify a date like {December 2020}

\begin{document}

\maketitle

\begin{problem}
Write the addition and multiplication tables for $\Z_2$.
\end{problem}

$$\begin{array}{c|cc}
    + & 0 & 1 \\
    \hline
    0 & 0 & 1 \\
    1 & 1 & 1 \\
  \end{array}, \quad
  \begin{array}{c|cc}
    \times & 0 & 1 \\
    \hline
    0      & 0 & 0 \\
    1      & 0 & 1
  \end{array}$$

\begin{problem}
Write the addition and multiplication tables for $\Z_3$.
\end{problem}

$$\begin{array}{c|ccc}
    + & 0 & 1 & 2 \\
    \hline
    0 & 0 & 1 & 2 \\
    1 & 1 & 2 & 0 \\
    2 & 2 & 0 & 1 \\
  \end{array}, \quad
  \begin{array}{c|ccc}
    \times & 0 & 1 & 2 \\
    \hline
    0      & 0 & 0 & 0 \\
    1      & 0 & 1 & 2 \\
    2      & 0 & 2 & 1 \\
  \end{array}$$

\begin{problem}
Write the addition and multiplication tables for $\Z_4$.
\end{problem}

$$\begin{array}{c|cccc}
    + & 0 & 1 & 2 & 3 \\
    \hline
    0 & 0 & 1 & 2 & 3 \\
    1 & 1 & 2 & 3 & 0 \\
    2 & 2 & 3 & 0 & 1 \\
    3 & 3 & 0 & 1 & 2 \\
  \end{array}, \quad
  \begin{array}{c|cccc}
    \times & 0 & 1 & 2 & 3 \\
    \hline
    0      & 0 & 0 & 0 & 0 \\
    1      & 0 & 1 & 2 & 3 \\
    2      & 0 & 2 & 0 & 2 \\
    3      & 0 & 3 & 2 & 1 \\
  \end{array}$$

\begin{problem}
Write the addition and multiplication tables for $\Z_6$.
\end{problem}
$$\begin{array}{c|cccccc}
    + & 0 & 1 & 2 & 3 & 4 & 5 \\
    \hline
    0 & 0 & 1 & 2 & 3 & 4 & 5 \\
    1 & 1 & 2 & 3 & 4 & 5 & 0 \\
    2 & 2 & 3 & 4 & 5 & 0 & 1 \\
    3 & 3 & 4 & 5 & 0 & 1 & 2 \\
    4 & 4 & 5 & 0 & 1 & 2 & 3 \\
    5 & 5 & 0 & 1 & 2 & 3 & 4 \\
  \end{array}, \quad
  \begin{array}{c|cccccc}
    \times & 0 & 1 & 2 & 3 & 4 & 5 \\
    \hline
    0      & 0 & 0 & 0 & 0 & 0 & 0 \\
    1      & 0 & 1 & 2 & 3 & 4 & 5 \\
    2      & 0 & 2 & 4 & 0 & 2 & 4 \\
    3      & 0 & 3 & 0 & 3 & 0 & 3 \\
    4      & 0 & 4 & 2 & 0 & 4 & 2 \\
    5      & 0 & 5 & 4 & 3 & 2 & 1 \\
  \end{array}$$

\begin{problem}
Suppose $[a], [b] \in \Z_5$ and $[a] \cdot [b] = [0]$. Is it necessarily true that either $[a] = [0]$ or $[b] = [0]$?
\end{problem}

Yes because we can see from its multiplication chart that the only products becoming $[0]$ come from factors of $[0]$. More generally any member of $[5] = 5x$ for some integer $x$. In the prime factorization of $5x$, the prime 5 has at least a power of 1. And if $5x = a \cdot b$ for some integers $a$ and $b$, it would require that either $a$ or $b$ have a factor of 5 which would require $[a] = 0$ or $[b] = 0$.

\begin{problem}
Suppose $[a], [b] \in \Z_6$ and $[a] \cdot [b] = [0]$. Is it necessarily true that either $[a] = [0]$ or $[b] = [0]$? What if $[a], [b] \in \Z_7$?
\end{problem}

No, we saw from its multiplication table that nonzero classes could multiply to $[0]$ such as $[2] \cdot [3] = [0]$. More generally, the prime factorization of any element in $[0]$ looks like $2\cdot3\cdot x$ for some integer $x$. This means if $2\cdot 3 \cdot x = a \cdot b$ for some integers $a$ and $b$, it could be that $a$ provides the factor of 2 and $b$ provides the factor of 3. For example, $a = 2, b = 3, x = 1$.

If we're in $\Z_7$ we might observe that we're in a field which would require that $[a] = [0]$ or $[b] = [0]$. If we don't make that observation again we could argue using prime decomposition. The prime factorization of any number in $[0] = 7n$ for some integer $n$. If $7n = a \cdot b$ for integers $a$ and $b$, then either $a$ or $b$ must have a factor of 7, since 7 is prime and cannot be the product of smaller factors. And if $a$ or $b$ has a factor of 7 then it is a multiple of 7 and belongs to $[0]$.

\begin{problem}
Do the following calculations in $\Z_9$, in each case expressing your answer as $[a]$ with $0 \leq a \leq 8$.

\begin{enumerate}
  \item $[8] + [8]$
  \item $[24] + [11]$
  \item $[21] \cdot [15]$
  \item $[8] \cdot [8]$
\end{enumerate}
\end{problem}

\begin{enumerate}
  \item $[8] + [8] = [16] = [7]$
  \item $[24] + [11] = [35] = [8]$
  \item $[21] \cdot [15] = [3] \cdot [6] = [18] = [0]$
  \item $[8] \cdot [8] = [64] = [1]$
\end{enumerate}

\begin{problem}
Suppose $[a], [b] \in \Z_n$ and $[a] = [a']$ and $[b] = [b']$. Alice adds $[a]$ and $[b]$ as $[a] + [b] = [a + b]$. Bob adds them as $[a'] + [b'] = [a' + b']$. Show that their answers $[a + b]$ and $[a' + b']$ are the same.
\end{problem}

Since $a \in [a]$ we can write it as $a = xn + c$ where $c$ is the remainder. Likewise $a' \in [a]$ so $a' = x'n + c$. Also write $b = yn + d, b' = y'n + d$.

Now $a + b = (x + y)n + (c + d)$ so it is in the equivalence class $[c + d]$. Likewise $a' + b' = (x' + y')n + (c + d)$ so it is in the equivalence class $[c + d]$. Therefore Alice and Bob's answers are the same.
\end{document}
