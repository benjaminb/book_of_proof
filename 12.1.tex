\documentclass{article}
\usepackage{amsmath} % for advanced math environments
\usepackage{amsfonts} % for math fonts
\usepackage{amssymb} % for math symbols
\usepackage{amsthm} % for theorems and proofs
\usepackage{mathtools} % for mathematical tools
\usepackage{mathrsfs} % for script-like fonts in math
\usepackage{bm} % for bold math symbols
\usepackage{bbm} % for "blackboard-style" characters in math
\usepackage{graphicx} % for including graphics
\usepackage{hyperref} % for including hyperlinks
\usepackage{tcolorbox}
\usepackage{tikz}
\tcbuselibrary{theorems, breakable}
\usepackage{xcolor}
\usepackage[margin=1in]{geometry}

\newcommand{\C}{\mathbb{C}}
\newcommand{\N}{\mathbb{N}}
\newcommand{\Q}{\mathbb{Q}}
\newcommand{\R}{\mathbb{R}}
\newcommand{\Z}{\mathbb{Z}}
\newcommand{\pset}{\mathscr{P}}
\DeclareMathOperator{\lcm}{lcm}

% Define a shortcut for \begin{bmatrix} and \end{bmatrix}
\newcommand{\bmat}[1]{\begin{bmatrix}#1\end{bmatrix}}
\newcommand{\cmat}[1]{\begin{pmatrix}#1\end{pmatrix}}

\newtcolorbox[auto counter]{problem}%
{
    breakable,
    colback=cyan!5,
    colframe=cyan!35!black,
    fonttitle=\bfseries,
    title=Problem~\thetcbcounter,
}

\newtcolorbox{solution}[1]
{
    breakable,
    colback=red!5,
    colframe=red!75!black,
    fonttitle=\bfseries,
    title=Solution: #1,
}

% Title
\title{12.1 Functions}
\author{Benjamin Basseri}


\begin{document}

\maketitle

\begin{problem}
Suppose $A = \{0, 1, 2, 3, 4\}, B = \{2, 3, 4, 5\}$ and $f = \{(0, 3), (1, 3), (2, 4), (3, 2), (4, 2)\}$. State the domain and range of $f$. Find $f(2)$ and $f(1)$.
\end{problem}

The domain of $f$ is the set of all first coordinates in $f$'s set representation, which is $A$. The range of $f$ is another term for the image, meaning the set of all second coordinates in $f$'s set representation, which is $\{2, 3, 4\}$. Scanning through the set $f$ we find $f(2) = 4, f(1) = 3$.

\begin{problem}
Suppose $A = \{a, b, c, d\}, B = \{2, 3, 4, 5, 6\}$ and $f = \{(a, 2), (b, 3), (c, 4), (d, 5)\}$. State the domain and range of $f$. Find $f(b)$ and $f(d)$.
\end{problem}

The domain is $A$, the set of all first coordinates. The range is $\{2, 3, 4, 5\}$ which is not quite $B$ but a subset of $B$. Scanning the set $f$ we see $f(b) = 3, f(d) = 5$.

\begin{problem}
There are four different functions $f: \{a, b\} \to \{0, 1\}$. List them
\end{problem}
$$\begin{array}{c|c}
    x & f(x) \\
    \hline
    a & 0    \\
    b & 0    \\
  \end{array}, \quad
  \begin{array}{c|c}
    x & f(x) \\
    \hline
    a & 1    \\
    b & 1    \\
  \end{array}, \quad
  \begin{array}{c|c}
    x & f(x) \\
    \hline
    a & 0    \\
    b & 1    \\
  \end{array}, \quad
  \begin{array}{c|c}
    x & f(x) \\
    \hline
    a & 1    \\
    b & 0    \\
  \end{array}
$$

\begin{problem}
There are eight different functions $f: \{a, b, c\} \to \{0, 1\}$. List them
\end{problem}

Constant 0, constant 1, three functions where exactly one element in the domain maps to 0 while the others map to 1, and three functions where exactly two elements in the domain map to 0 and exactly one element maps to 1.

\begin{problem}
Give an example of a relation from $\{a, b, c, d\}$ to $\{d, e\}$ that is not a function.
\end{problem}

For a relation to *not* be a function it must fail the `vertical line test', which in the set view means we can't have two ordered pairs with the same first coordinate. So for example:
$$R = \{(a, d), (a, e)\}$$
is a relation from the first set to the second but $a$ relates to two different elements. As a function this would not be `well-defined'.

\begin{problem}
Suppose $f: \Z \to \Z$ is defined as $f = \{(x, 4x + 5): x \in \Z\}$. State the domain, codomain, and range of $f$. Find $f(10)$.
\end{problem}

The domain is $\Z$ since the first coordinate of all pairs in $f$ come directly from $\Z$. The codomain is also $\Z$ since for any integer $x$, $4x + 5$ will also be an integer. The range is $\{\ldots, -3, 1, 5, 9, \ldots\}$ or $\{4x + 5: x \in \Z\}$. By plugging in 10 we get $f(10) = 45$.

\begin{problem}
Consider the set W$f = \{(x, y) \in Z \times \Z: 3x + y = 4\}$. Is this a function from $\Z$ to $\Z$? Explain.
\end{problem}

It's certainly a relation from $\Z$ to $\Z$ since $f \subseteq \Z \times \Z$. For it to be a function, $f$ must have a unique $y$ for any $x$ in a pair $(x, y)$. I claim this is true for $f$. To prove this we'll use the 'set up two things and prove they're equal' technique.

Suppose $(x, y) \in f$. Then $3x + y = 4$ or in terms of $y$ we have $y = 4 - 3x$. Now if there is another element $(x, z) \in f$ then $z = 4 - 3x$ which means $z = y$.

\begin{problem}
Consider the set $f = \{ (x, y) \in \Z \times \Z: x + 3y = 4\}$. Is this a function from $\Z$ to $\Z$? Explain.
\end{problem}

Yes, for precisely the same reasons as before. Set $f$ is a subset of $\Z \times \Z$ so it's a relation. By the same equality argument as before we can show that $f$ is well-defined. Therefore it's a function.

\begin{problem}
Consider the set $f = \{(x^2, x): x \in \R\}$. Is this a function from $\R$ to $\R$? Explain.
\end{problem}

No, because it's not well-defined. For example $(4, 2)$ and $(4, -2)$ satisfy the specification for $f$ and so are both members.

\begin{problem}
Consider the set $f = \{(x^3, x): x \in \R\}$. Is this a function from $\R$ to $\R$? Explain.
\end{problem}

Yes, this is a function. Another way of looking at the relation is that the second coordinate is the cube root of the first. Since every $x \in \R$ is a cube of some number, the first coordinate (domain) is all $\R$. And since every number has a unique cube root, there is only one possible $x$ for any $x^3$, making $f$ well-defined.

\begin{problem}
Is the set $\theta = \{(X, |X|): X \subseteq \Z_5\}$ a function? If so, what is its domain and range?
\end{problem}

If we consider $\Z_5 = \{[0], [1], [2], [3], [4]\}$ then $\Z_5$ is finite (even though its members aren't) and the subsets of $\Z_5$ are finite with well-defined size. So $\theta$ is a valid function. It's domain is the equivalence classes modulo 5 and its range is $\{0, 1, 2, 3, 4, 5\}$.

\begin{problem}
Is the set $\theta = \{\left((x, y), (3y, 2x, x + y)\right): x, y \in \R\}$ a function? If so, what is its domain and range? What can be said about the codomain?
\end{problem}

We can see that $\theta$ associates points in $\R^2$ to points in $\R^3$, and each second coordinate results from three linear component functions, therefore $f$ is a subset of $\R^2 \times \R^3$ and is well-defined, so it's a function.
\end{document}