\documentclass{article}
\usepackage{amsmath} % for advanced math environments
\usepackage{amsfonts} % for math fonts
\usepackage{amssymb} % for math symbols
\usepackage{amsthm} % for theorems and proofs
\usepackage{mathtools} % for mathematical tools
\usepackage{mathrsfs} % for script-like fonts in math
\usepackage{bm} % for bold math symbols
\usepackage{bbm} % for "blackboard-style" characters in math
\usepackage{graphicx} % for including graphics
\usepackage{hyperref} % for including hyperlinks
\usepackage{tcolorbox}
\usepackage{tikz}
\tcbuselibrary{theorems, breakable}
\usepackage{xcolor}
\usepackage[margin=1in]{geometry}

\newcommand{\C}{\mathbb{C}}
\newcommand{\N}{\mathbb{N}}
\newcommand{\Q}{\mathbb{Q}}
\newcommand{\R}{\mathbb{R}}
\newcommand{\Z}{\mathbb{Z}}
\newcommand{\pset}{\mathscr{P}}
\DeclareMathOperator{\lcm}{lcm}

% Define a shortcut for \begin{bmatrix} and \end{bmatrix}
\newcommand{\bmat}[1]{\begin{bmatrix}#1\end{bmatrix}}
\newcommand{\cmat}[1]{\begin{pmatrix}#1\end{pmatrix}}

\newtcolorbox[auto counter]{problem}%
{
    breakable,
    colback=cyan!5,
    colframe=cyan!35!black,
    fonttitle=\bfseries,
    title=Problem~\thetcbcounter,
}

\newtcolorbox{solution}[1]
{
    breakable,
    colback=red!5,
    colframe=red!75!black,
    fonttitle=\bfseries,
    title=Solution: #1,
}

% Title
\title{Your Document Title}
\author{Benjamin Basseri}


\begin{document}

\maketitle

\begin{problem}
Let $A = \{1, 2, 3, 4, 5, 6\}$, and consider the following equivalence relation on $A$:
$$R = \{(1, 1), (2, 2), (3, 3), (4, 4), (5, 5), (6, 6),$$
$$(2, 3), (3, 2), (4, 5), (5, 4), (4, 6), (6, 4), (5, 6), (6, 5)\}$$

List the equivalence classes of $R$.
\end{problem}

We see that 1 doesn't have any relations except to itself, so we have $[1] = \{1\}$. However 2 and 3 have a relation, and nothing else relates to 2 or 3 but each other. So we have a second equivlance class $[2] = \{2, 3\}$. Finaly, 4, 5, and 6 all have relations with each other so our last equivalence class is $[4] = \{4, 5, 6\}$.

\begin{problem}
Let $A = \{a, b, c, d, e\}$. Suppose $R$ is an equivalence relation on $A$. Suppose $R$ has two equivalence classes. Also $aRd, bRc$, and $eRd$. Write out $R$ as a set.
\end{problem}

Since $R$ is an equivalence relation it must be reflexive and contain the diagonal of $A$. It must also be transitive so $a, d$, and $e$ must all be in the same class. We see none of those elements have a relationship with $b$ or $c$, although $b$ and $c$ relate to each other. So the two equivalence classes are $\{a, d, e\}$ and $\{b, c\}$. Writing out $R$:
$$R = \{(a, a), (b, b), (c, c), (d, d), (e, e), $$
$$(a, d), (d, a), (a, e), (e, a), (d, e), (e, d), (b, c), (c, b)\}$$

\begin{problem}
Let $A = \{a, b, c, d, e\}$. Suppose $R$ is an equivalcence relation on $A$. Suppose $R$ has three equivalence classes. Also $aRd$ and $bRc$. Write out $R$ as a set.
\end{problem}

First let's identify the equivalence classes. There is a clas containing $a$ and $d$, and two more classes. If $b$ and $c$ are in the same class as $a$ and $d$ then the only remaining element to form a new class is $e$ and there would only be two equivlance classes. Therefore $b$ and $c$ must be in a different equivalcence class and $e$ must be a singleton class for the third one. This makes the classes $\{a, d\}, \{b, c\}, \{e\}$
$$R = \{(a, a), (b, b), (c, c), (d, d), (e, e), $$
$$(a, d), (d, a), (b, c), (c, b)\}$$

\begin{problem}
Let $A= \{a, b, c, d, e\}$. Suppose $R$ is an equivalence relation on $A$. Suppose that $aRd$ and $bRc$, $eRa$ and $cRe$. How many equivalence classes does $R$ have?
\end{problem}

Since transitivity and symmetry holds in each equivalence class, $aRd$ implies $a$ and $d$ are in the same class, $eRa$ implies $e$ is also in the class, $cRe$ means $c$ is in the class too, and $cRb$ means $b$ is in the class. So all elements are in the class and there is just one big equivalence class.

\begin{problem}
There are two difference equivalence relations on the set $A = \{a, b\}$. Describe them.
\end{problem}

Either we have one big clump and $R = \{(a, a), (b, b), (a, b), (b, a)\}$ or we have complete separation and $R = \{(a, a), (b, b)\}$.

\begin{problem}
There are five different equivalence relations on the set $A = \{a, b, c\}$. Describe them all.
\end{problem}

We could have either one, two, or three equivalence classes. There's only one may to have one equivalence class and that's the one big clump relation $R = A^2$. There's also only one way to have three equivalence classes, and that's with complete separation, $R = \{(a, a), (b, b), (c, c)\}$. Then there are $\binom{3}{2} = 3$ ways to choose 2 of the 3 elements to be equivalent and these are the remaining three equivalence relations:
$$[1] = \{a, b\}, [2] = \{c\}$$
$$[1] = \{a, c\}, [2] = \{b\}$$
$$[1] = \{b, c\}, [2] = \{a\}$$

\begin{problem}
Define a relation $R$ on $\Z$ as $xRy$ if and only if $3x - 5y$ is even. Prove $R$ is an equivalence relation. Describe its equivalence classes.
\end{problem}

There will be two equivalence classes total: one for elements $(x, y)$ where $3x - 5y$ are even and the other for when the difference is odd. Now let's check the relation properties. To see if it's reflexive we test $xRx$ for any integer $x$, which exists if $3x - 5x = -2x$ is even and it is. Therefore $R$ is reflexive. Now we test symmetry: suppose $xRy$ so $3x - 5y$ is even. The only way this difference could be even is if $3x$ and $5y$ have the same parity, otherwise it would be (odd) $-$ (even) or (even) $-$ (odd) which would be odd. Now consider $3y - 5x$. This differs from the first expression by $-8x$, which must be even, and $8x$ which must also be even. That is
$$3y - 5x = 3x - 5y + 2(4y - 4x)$$
This means $3y - 5x$ is an even number plus an even number, which is even. Therefore we have $yRx$ and the relation is symmetric.

Finally to test transitivity suppose we have $xRy$ and $yRz$. Then $3x - 5y$ is even as is $3y - 5z$. Then $3x - 5y = 2a$ for some integer $a$ and $3y - 5z = 2b$ for some integer $b$. By multiplying the first equation by 3 and second equation by 5, then adding them together, we obtain:
$$9x + 5z = 6a - 10b$$
If we subtract $6x$ and $10z$ from both sides we get
$$9x + 5z -6x - 10z = 6a - 10b - 6x - 10z$$
$$\implies 3x - 5z = 2(3a - 5b - 3x - 5z)$$
which is even.


\begin{problem}
Define a relation $R$ on $\Z$ as $xRy$ if and only if $x^2 + y^2$ is even. Prove $R$ is an equivalence relation. Describe the equivalence classes.
\end{problem}

There are two classes: one where the sum is even and one where the sum is odd. Reflexive:
$$x^2 + x^2 = 2x^2 \text{ (even) } \implies xRx$$

We also have symmetry since if $x^2 + y^2$ is even then by commutative addition this equals $y^2 + x^2$, which will also be even. If we have $xRy$ and $yRz$ then I claim $x^2$, $y^2$ and $z^2$ all have the same parity. First, since $x^2 + y^2$ is even this means either both $x^2$ and $y^2$ are even or they're both odd. Then for $y^2 + z^2$ to be even, $z^2$ must have the same parity as $y^2$. Then $z^2$ has the same parity as $x^2$ (as parity is itself an equivalence relation and hence transitive) and $x^2 + z^2$ is even, giving us $xRy \land yRz \implies xRz$.

\begin{problem}
Define a relation $R$ on $\Z$ as $xRy$ if and only if $4 \mid (x + 3y)$. Prove $R$ is an equivalence relation. Describe its equivalcence classes.
\end{problem}

There are two equivalence classes: $\{(x, y): 4 \mid (x + 3y) \}$ and its complement (those $x, y$ such that $4 \not\mid (x + 3y)$).

The relation is reflexive since for any $x$ have $x + 3x = 4x$ which is divisble by 4. The relation is symmetric: suppose we have $xRy$ so 4 divides $x + 3y$. Then $x + 3y = 4a$ for some integer $a$. Multiplying the equation by 3 gives $3x+ 9y = 12a$, and subtracting $8y$ gives $3x + y = 4(3a - 2y)$. Therefore 4 divides $y + 3x$ and we have $yRx$. The relation is transitive: suppose $xRy$ and $yRz$. Then $x + 3y = 4a$ for some $a$ and $y + 3z = 4b$ for some $b$. Multiply that second equation by 3 and subtract it from the first equation to obtain:
$$x + 3y - 3y - 9z = x - 9z = 4a - 12b$$
$$\implies x + 3z = 4a - 12b + 12z = 4(a - 3b + 3z)$$
which is divisible by 4.

\begin{problem}
Suppose $R$ and $S$ are two equivalence relations on a set $A$. Prove that $R \cap S$ is also an equivalence relation.
\end{problem}

Since both $R$ and $S$ are equivalence relations. Then they both contain the diagonal of $A \times A$ and so does their intersection. Therefore $R \cap S$ is reflexive. Now suppose there is an $(x, y)$ in the intersection. Then $(x, y)$ is in both $R$ and $S$ and by their symmetry, $(y, x)$ is also in both $R$ and $S$. Therefore $(y, x)$ is in $R \cap S$ and the intersection is symmetric. Likewise, if $(x, y), (y, z)$ is in the intersection then both those relationships are in $R$ and $S$ as well. Since those relations are transitive they both contain $(x, z)$ and $(x, z)$ is in the intersection, so $R \cap S$ is transitive.

\begin{problem}
Prove or disprove: If $R$ is an equivalence relation on an infinite set $A$, then $R$ has infinitelyi many equivalence classes.
\end{problem}

This is false. Consider the same parity relation on $\Z$, which is an infinite set. There are only two equivalence classes: $\{\ldots, -4, -2, 0, 2, -4, \ldots \}$ and $\{\ldots, -3, -1, 1, 3, \ldots\}$.

\begin{problem}
Prove or disprove: If $R$ and $S$ are two equivalence relations on a set $A$, then $R \cup S$ is also an equivalence relation on $A$.
\end{problem}

$R \cup S$ will be reflexive since both $R$ and $S$ contain the diagonal, so their union certainly contains it as well. Now suppose $(x, y)$ is in the union. Then it is either a member of $R$ or $S$ (or both). Then $(y, x)$ is also in the base set hence it is in the union. Likewise if $(x, y)$ and $(y, z)$ are in the union, it is not necessarily true that the transitive relationship $(x, z)$ is in the union. For example it could be that $(x, y)$ in the union came from $R$ and has not transitive relationships (such as $R$ is just the diagonal, $(x, y)$, and $(y, x)$) and $(y, z)$ originates in $S$ with the same setup that $S$ is just the diagonal, $(y, z)$, and $(z, y)$. Then $(x, z)$ is not in either $R$ or $S$ and not in the union, so the union is not transitive.

\begin{problem}
Suppose $R$ is an equivalence relation on a finite set $A$, and every equivalence class has the same cardinality $m$. Express $|R|$ in terms of $m$ and $|A|$.
\end{problem}

Each equivalence class has $m$ elements which means it corresponds to $m^2$ pairs in $R$, since each element in the class has a relation to every other element including itself. There are $|A|/m$ total equivalence classes because equivalence classes partition the set. Therefore $|R| = m^2 |A| / m = m|A|$.

\begin{problem}
Suppose $R$ is a reflexive and symmetric relation on a finite set $A$. Define a relation $S$ on $A$ by declaring $xSy$ if and only if for some $n \in \N$ there are elements $x_1, x_2, \ldots, x_n \in A$ satisfying $xRx_1, x_1 R x_2, \ldots, x_{n-1} R x_n$, and $x_n R y$. Show that $S$ is an equivalence relation and $R \subseteq S$. Prove that $S$ is the unique smallest equivalence relation on $A$ containing $R$.
\end{problem}

First let's show $S$ is an equivalence relation. $S$ has relations $xSy$ if and only if a finite chain of relations can be made in $R$ from $x$ to $y$. For reflexivity, can we make a `chain' from $x$ to itself in $R$? Even if we require that $n$ be at least 1, then we need some $x_1$ such that $xRx_1, x_1Rx$. Of course, nothing prevents us from using $x$ itself as $x_1$, so we check if $xRx$ is a relation and since $R$ is reflexive, it has all self-relations. Therefore $S$ is reflexive.

$S$ is symmetric because $xSy$ implies $xRx_1, \ldots, x_n R y$ and since $R$ is symmetric, the reverse relations exist in $R$ as well: $yRx_n, x_n R x_{n-1}, \ldots , x_1 R x$. This implies $ySx$ and $S$ is symmetric.

$S$ is transitive for similar reasons. Suppose $xSy, ySz$. Then in $R$ we have:
$$xRa_1, xRa_2, \ldots , xRa_n, a_n R y$$
$$yRb_1, yRb_2, \ldots yRb_m, b_m R z$$

This means we have a finite chain of relations from $x$ to $z$ in $R$, which means we have $xSz$ and $S$ is transitive. Therefore $S$ is an equivalence relation.

To prove $R \subset S$, for any $xRy$ since $R$ is symmetric we could form the chain $xRy, yRx, xRy$ (assuming we require $n \geq 1$) which implies $xSy$. Therefore any $(x, y) \in R \implies (x, y) \in S$ and $S$ contains $R$.

To show $S$ is the unique smallest equivalence relation on $A$ containing $R$, consider taking $R$ which is already reflexive and symmetric and closing it under transitive relations to form an equivalence relation $S$. Then $(x, x_1)$ and $(x_1, x_2) \in S$ would require $(x, x_2) \in S$. If there's an $(x_2, x_3) \in R$ then we would also have $(x, x_3) \in S$. Continuing this way we see that any equivalence relation containin $R$ must have the chaining property. This means any equivalence relation containing $R$ must contain $S$, so $S$ is the smallest possible set that's an equivalence relation containing $R$. And since sets are defined by their membership $S$ is unique.

\begin{problem}
Suppose $R$ is an equivalence relation on a set $A$, with four equivalence classes. How many different equivalence relations $S$ on $A$ are there for which $R \subset S$?
\end{problem}

To contain $R$ any other relation $S$ would have to keep all the equivalence relations already present in $R$. These are partitioned into 4 classes, call them $[1], [2], [3], [4]$. So the only thing we can do is 'merge' one or more equivalency classes. Let's check that this preserves $R$.

Suppose to form $S$ we merge classes $[1]$ and $[2]$. For any $a, b \in [1]$ this indicates $(a, b), (b, a) \in R$. Merging the classes still maintains $a, b \in [1\mid 2]$, so $(a, b), (b, a) \in S$. In fact all relations stemming from equivalence class 1 or 2 are still preserved in $S$. Now suppose $a \in [1], b \in [2]$. If we merge these two classes this requires $(a, b), (b, a) \in S$, and any other transitive relations that need to be present for the elements of $[1]$ and $[2]$ to be considered equivalent. These are relations not present in $R$, for if they were $[1]$ and $[2]$ would already have the same elements and would in fact be the same subset of $A$. Therefore merging equivalence classes still contains all previous relations in $R$ while creating new ones, and the merges would all be distinct from each other since equivalency classes contain distinct elements.

So all this boils down to how many and which equivalency classes we choose to merge in order to create an equivalency relation containing $R$. In other words, it's the number of ways to partition a set of 4 objects, which is 15.

\begin{problem}
Show that the relation $\doteq$ defined on page 213 is transitive.
\end{problem}

This relation states $\frac{a}{b}\doteq\frac{c}{d}$ when $ad = bc$. This is reflexive, since $ab = ba \implies \frac{a}{b}\doteq\frac{a}{b}$. This is also clearly symmetric since $ad = bc \implies bc = ad$. For transitivity, suppose $\frac{a}{b}\doteq\frac{c}{d}$ and $\frac{c}{d}\doteq\frac{e}{f}$. Then $ad = bc$ and $cf = de$. Take $cf = de$, multiply the left by $ad$ and the right by $bc$ (these are in fact the same value from the first relation above, so this multiplication preserves equality). Then we get $adcf = bcde$. Canceling $c$ and $d$ from both sides leaves $af = be$, implying $\frac{a}{b}\doteq\frac{e}{f}$.


\end{document}