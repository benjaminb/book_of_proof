\documentclass{article}
\usepackage{amsmath} % for advanced math environments
\usepackage{amsfonts} % for math fonts
\usepackage{amssymb} % for math symbols
\usepackage{amsthm} % for theorems and proofs
\usepackage{mathtools} % for mathematical tools
\usepackage{mathrsfs} % for script-like fonts in math
\usepackage{bm} % for bold math symbols
\usepackage{bbm} % for "blackboard-style" characters in math
\usepackage{graphicx} % for including graphics
\usepackage{hyperref} % for including hyperlinks
\usepackage{tcolorbox}
\usepackage{tikz}
\tcbuselibrary{theorems, breakable}
\usepackage{xcolor}
\usepackage[margin=1in]{geometry}

\newcommand{\C}{\mathbb{C}}
\newcommand{\N}{\mathbb{N}}
\newcommand{\Q}{\mathbb{Q}}
\newcommand{\R}{\mathbb{R}}
\newcommand{\Z}{\mathbb{Z}}
\newcommand{\pset}{\mathscr{P}}
\DeclareMathOperator{\lcm}{lcm}

% Define a shortcut for \begin{bmatrix} and \end{bmatrix}
\newcommand{\bmat}[1]{\begin{bmatrix}#1\end{bmatrix}}
\newcommand{\cmat}[1]{\begin{pmatrix}#1\end{pmatrix}}

\newtcolorbox[auto counter]{problem}%
{
    breakable,
    colback=cyan!5,
    colframe=cyan!35!black,
    fonttitle=\bfseries,
    title=Problem~\thetcbcounter,
}

\newtcolorbox{solution}[1]
{
    breakable,
    colback=red!5,
    colframe=red!75!black,
    fonttitle=\bfseries,
    title=Solution: #1,
}

% Title
\title{Your Document Title}
\author{Your Name}
\date{\today} % or specify a date like {December 2020}

\begin{document}

\maketitle

\section{Use the method of proof by contradiction to prove the following statements.}

\begin{problem}
Suppose $n \in \Z$. If $n$ is odd then $n^2$ is odd.
\end{problem}

\begin{solution}{Use proof by contradiction.}
    Suppose $n \in \Z$ is odd but $n^2$ is not odd. Then $n^2$ is even and equals $2k$ for some integer $k$. But since $n$ is odd it equals $2j + 1$ for some integer $j$, and
    $$n^2 = (2j + 1)^2 = 4j^2 + 4j + 1 = 2(2j^2 + 2j) + 1.$$
    This means $n^2$ has the form of an odd integer, which contradicts the assumption it was even.

    Although the book asks us to make this a proof by contradiction, it would be simpler as a direct proof: $n = 2j + 1$ for some $j \in \Z$, then $n^2 = 2(2j^2 + 2j) + 1$, which is odd.
\end{solution}

\begin{problem}
    Suppose $n \in \Z$. If $n^2$ is odd then $n$ is odd.
\end{problem}

\begin{solution}{Use proof by contradiction}
    Let's assume towards a contradiction that $n^2$ is odd but $n$ is not odd. Since $n$ is even it equals $2k$ for some $k \in \Z$, but then $n^2 = 4k^2 = 2(2k^2)$ which is even. This contradicts the assumption that $n^2$ was odd.
\end{solution}

\begin{problem}
    Prove that $\sqrt[3]{2}$ is irrational.
\end{problem}

\begin{solution}{Use proof by contradiction}
    Similar to the proof that $\sqrt{2}$ is irrational, let $\sqrt[3]{2} = a/b$ with $a/b$ in lowest terms. This means $a$ and $b$ have no common factors. Cube this to get 
    $$2 = a^3 / b^3 \implies 2b^3 = a^3.$$
    This means that $a^3$ is even and what's more, it must be that $a$ is even. Otherwise $a^3$ would be the product of three odds which is odd. Since $a$ is even it has a factor of $2$ and therefore $a^3$ must have a factor of 8. So we can write $a^3 = 8k$ for some integer $k$. This gives
    $$2b^3 = 8k \implies b^3 = 4k.$$
    Now we see $b^3$ is even as well and by the same argument, $b$ must be even and have a factor of 2. So both $a$ and $b$ must have a factor of two, but this contradicts the assumption that $a/b$ was in lowest terms.
\end{solution}

\begin{problem}
    Prove that $\sqrt{6}$ is irrational.
\end{problem}
\begin{solution}{Use proof by contradiction}
    Again similar to the proof that $\sqrt{2}$ is irrational, assume towards a contradiction that $\sqrt{6} = a/b$ with $a/b$ in lowest terms. Then $6 = a^2/b^2$ and $6b^2 = a^2$. This means both 2 and 3 divide $a^2$ since it is a multiple of 6. Since both 2 and 3 are primes, if they are factors of $a^2$ they must be factors of $a$ as well. And since 2 and 3 both divide $a$, $2^2$ and $3^2$ must divide $a^2$. So $a^2 = 2^2 \cdot 3^3 \cdot k$ for some integer $k$. We can then write $b^2$ in terms of $k$:
    $$6b^2 = 2^2 \cdot 3^2 \cdot k \implies b^2 = 2\cdot 3 \cdot k.$$
    By the same logic as above, 2 and 3 must divide $b$. But $a$ also had those factors, and this contradicts the assumption that $a/b$ was in lowest terms.
\end{solution}

\begin{problem}
    Prove that $\sqrt{3}$ is irrational.
\end{problem}
\begin{solution}{Use proof by contradiction}
    Again similar to the proof that $\sqrt{2}$ is irrational, assume towards a contradiction that $\sqrt{3} = a/b$ with $a$ and $b$ in lowest terms. 

    $$\sqrt{3} = \frac{a}{b} \implies 3 = \frac{a^2}{b^2} \implies 3b^2 = a^2.$$
    This shows 3 divides $a^2$, and since 3 is prime if it factors $a^2$ it must factor $a$ as well. Then 9 divides $a^2$ and we can write $a^2 = 9k$ for some integer $k$. Substituting this in the last equality above, $3b^2 = 9k$, simplifies to $b^2 = 3k$. This shows that 3 divides $b$ as well. But if $a$ and $b$ both have a factor of 3 this contradicts the assumption that $a/b$ was in lowest terms.

\end{solution}

\begin{problem}
    If $a, b \in\Z$, then $a^2 - 4b - 2 \neq 0$.
\end{problem}
\begin{solution}{Use divisibility properties to derive a contradiction}
   Assume towards a contradiction that $a^2 - 4b -2 = 0$. Rewrite this as $a^2 = 4b + 2$. Notice that $a^2$ is even so 2 must divide it. But since 2 is prime, if it divides $a^2$ then it must divide $a$ as well. So in fact $a^2$ is a multiple of 4. However, 4 does not divide $4b + 2$, since $4b$ will be a multiple of 4 for any integer $b$, dividing $4b + 2$ by 4 will leave a remainder of 2. So $4 \mid a^2$ but $4 \nmid 4b + 2$, contradicting the equality $a^2 = 4b + 2$.
\end{solution}

\begin{problem}
    If $a, b \in \Z$, then $a^2 - 4b - 3 \neq 0$.
\end{problem}
\begin{solution}{Use parity to derive a contradiction}

    
\end{solution}

\begin{problem}
    Suppose $a, b, c \in \Z$. If $a^2 + b^2 = c^2$, then $a$ or $b$ is even.
\end{problem}
\begin{solution}{Use divisibility properties to derive a contradiction}
    Assume towards a contradiction that $a^2 + b^2 = c^2$ and that both $a$ and $b$ are odd. Write $a = 2k + 1$ and $b = 2j + 1$. This makes:
    $$4k^2 + 4k + 1 + 4j^2 + 4j + 1 = c^2$$
    $$4(k^2 + j^2 + k + j) + 2 = c^2.$$

    Next, $c^2$ must be even as the sum of two odds. This means $c$ must be even and divisible by 2, so $c^2$ has a factor of 4, making it $4z$ for some $z$. Therefore we have:
    $$4(k^2 + j^2 + k + j) + 2 = 4z$$

    Now note that the RHS is divisible by 4 but the LHS is not: it is a multiple of 4 plus 2. So for any integer values of $j, k$, dividing the LHS by 4 will leave a remainder of 2. This contradicts the notion that the two sides are equal.
\end{solution}

\begin{problem}
    Suppose $a, b, \in \R$. If $a$ is rational and $ab$ is irrational, then $b$ is irrational.
\end{problem}
\begin{solution}{Use proof by contradiction}
    Suppose $a$ is rational, $ab$ is irrational, but $b$ is rational. Since both $a, b \in \Q$, $a = c/d$ and $b = x/y$ for some $c, d, x, y \in \Z$. Then $ab = cx/yz$ which is rational, contradicting the assumption $ab$ is irrational.
\end{solution}

\begin{problem}
    There exist no integers $a$ and $b$ for which $21a + 30b = 1$. 
\end{problem}
\begin{solution}{Use proof by contradiction}
    Assume towards a contradiction that integers $a, b$ exist such that $21a + 30b = 1$. Rewrite as:
    \begin{align*}
        3(7a + 10b) &= 1 \\
        \implies 7a + 10b &= \frac{1}{3} \\
    \end{align*}
    We $7a + 10b$  equals 1/3, contradicting the fact that it should be an integer, as it is the sum of integers.
\end{solution}

\begin{problem}
    There exist no integers $a$ and $b$ for which $18a + 6b = 1$.
\end{problem}
\begin{solution}{Use proof by contradiction}
    Assume towards a contradiction that there are integers $a, b$ such that $18a + 6b = 1$. Factoring out a 6 gives $6(3a + b) = 1$ and $3a + b = 1/6$. This contradicts the fact that $3a + b$ should be an integer.
\end{solution}

\begin{problem}
    For every positive $x \in \Q$, there is a positive $y \in \Q$ for which $y < x$.
\end{problem}
\begin{solution}{Use proof by contradiction}
    Assume towards a contradiction that there exists a positive rational $x$ such that for any other positive rational $y$, $y \geq x$. In other words, $x$ is a minimum in the set of positive rationals. Since $x$ is rational and positive write it as $a/b$ with both $a$ and $b$ positive. Then let $y = \frac{a}{2b}$. This $y$ is also a positive rational but less than $x$, contradicting the assumption that $x$ was a minimum of positive rationals.
\end{solution}

\begin{problem}
    For every $x \in [\pi/2, \pi], \sin x - \cos x \geq 1$.
\end{problem}
\begin{solution}{Use trig functions' sign behavior in quadrant II to derive a contradiction}
    Assume towards a contradiction that $\sin x - \cos x < 1$ over the interval $[\pi/2, \pi]$. First, let's square the inequality to make $\sin^2$ and $\cos^2$ appear, allowing us to make use of the Pythagorean theorem:
    \begin{align*}
       \sin x - \cos x &< 1 \\
       (\sin x - \cos x)^2 &< 1 \\
       \sin^2 x - 2\sin x \cos x + \cos^2 x &< 1 \\
       (\sin^2 x + cos^2 x) - 2\sin x \cos x &< 1 \\ 
       -2\sin x \cos x &< 0 \\
        \sin x \cos x &> 0
    \end{align*}
    Now thing about the sign of $\sin$ and $\cos$ over this interval. At $x = \pi / 2$, $\sin x \cos x = 0$, and it equals 0 when $x = \pi$ as well. So we have a contradiction on the interval's boundary. Inside the interval, $\sin$ is strictly positive and $\cos$ is strictly negative. Therefore their product must be negative, again contradicting the inequality. Since we can derive an inequality for any $x$ in the interval, it must be that $\sin x - \cos x \geq 1$ here. 
\end{solution}

\begin{problem}
    If $A$ and $B$ are sets, then $A \cup (B - A) = \varnothing$.
\end{problem}
\begin{solution}{Use direct proof}
    Although the book asks for proof by contradition here, there is a very straightforward direct proof:
    The set $B - A$ cannot have any members of $A$ in it. Therefore it cannot have any members in common with $A$, and $A \cap (B - A)$ must be empty.
\end{solution}

\begin{problem}
    If $b \in \Z$ and $b \nmid k$ for every $k \in \N$, then $b = 0$.
\end{problem}
\begin{solution}{Use proof by contradiction}
    Suppose $b\neq 0$. Then $b$ is either positive or negative. Take the case that $b$ is positive; then it's also a natural number. However we'd have $b \mid b$, which contradicts the assumption that $b$ does not divide any real number. Taking the case that $b$ is negative, the $-b$ is positive and a natural number. Again, $b \mid -b$ which contradicts the assumption that $b$ doesn't divide any naturals.
\end{solution}

\begin{problem}
    If $a$ and $b$ are positive real numbers, then $a + b \geq 2\sqrt{ab}$.
\end{problem}
\begin{solution}{Use proof by contradiction}
   Assume towards a contradiction that $a$ and $b$ are both postiive but $a + b < 2\sqrt{ab}$. Squaring both sides gives:
   \begin{align*}
        (a + b)^2 < 4ab \\
        a^2 + 2ab + b^2 < 4ab \\
        a^2 - 2ab + b^2 < 0 \\
        (a - b)^2 < 0
    \end{align*}
    Here we have a contradiction, since the square of a real number cannot be negative.
\end{solution}

\begin{problem}
    For every $n \in \Z, 4 \nmid (n^2 + 2)$.
\end{problem}
\begin{solution}{Use proof by contradiction with key fact about squares mod 4}
    Assume towards a contradiction that $n \in \Z$ and $4 \mid (n^2 + 2)$. Then $n^2 + 2 \equiv 0 \pmod{4}$, which means $n^2 \equiv 2 \pmod{4}$. But squared integers can only be 0 or 1 mod 4: an even number squared has a factor of 4 making it 0 mod 4 and an odd number $2j + 1$ squared is $4j^2 + 4j + 1$ which is 1 mod 4. Therefore we have a contradiction that $n^2$ is a squared integer but also equivalent to $2 \pmod{4}$.
\end{solution}

\begin{problem}
    Suppose $a, b \in \Z$. If $4 \mid (a^2 + b^2)$, then $a$ and $b$ are not both odd.
\end{problem}
\begin{solution}{Use proof by contradiction with key fact about squares mod 4}
    Assume towards a contradiction that $a$ and $b$ are both odd. Then $a^2$ and $b^2$ are both 1 mod 4. Adding together gives $a^2 + b^2 = 2 \pmod{4}$, which contradicts the assumption that $4 \mid (a^2 + b^2)$.
\end{solution}


\section{Prove the following statements using any method from Chapters 4, 5, or 6.}

\begin{problem}
    The product of any five consecutive integers is divisible by 120. (For example, the product of 3, 4, 5, 6, and 7 is 2520, and $2520 = 120 \cdot 21$.)
\end{problem}
\begin{solution}{Use direct proof}
    Any 5 consecutive integers must contain a multiple of 5. Take our consecutive integers $n, n+1, \ldots, n + 4$, but now consider them modulo 5. Suppose $n \pmod{5} = 0$. Then $n$ is a multiple of 5. If $n \pmod{5} = 1$, then $n + 4$ will be the multiple of 5. Continuing this way, we can prove that if $n$ has some remainder 1 through 4 after division by 5, some other member of our sequence will have no remainder.\\

    By the same logic one of the sequence numbers must be divisible by 4, by 3, and by 2. Therefore The product $n(n+1)(n+2)(n+3)(n+4)$ can be written as $5(4)(3)(2)k$ for some integer $k$. Or in other words, $120k$, which is divisible by 120.

\end{solution}

\begin{problem}
    We say that a point $P = (x,y)$ in $\R^2$ is \textbf{rational} if both $x$ and $y$ are rational. More precisely, $P$ is rational if $P = (x, y) \in \Q^2$. An equation $F(x,y)=0$ is said to have a \textbf{rational point} if there exists $x_0, y_0 \in \Q$ such that $F(x_0, y_0) = 0$. For example, the curve $x^2 + y^2 - 1 = 0$ has rational point $(x_0, y_0)=(1,0).$ Shot that the curve $x^2 + y^2 - 3 = 0$ has no rational points.
\end{problem}
\begin{solution}{Use proof by contradiction}
    Assume towards a contradiction that the curve $x^2 + y^2 - 3 = 0$ has a rational point. Then there exists $x_0, y_0 \in \Q$ such that $x_0^2 + y_0^2 - 3 = 0$. Write $x_0 = a/b$ and $y_0 = c/d$ with $a, b, c, d \in \Z$ and both fractions in lowest terms. Then we have:
    \begin{align*}
        \frac{a^2}{b^2} + \frac{c^2}{d^2} - 3 &= 0 \\
        \frac{a^2}{b^2} + \frac{c^2}{d^2} &= 3\\
        a^2d^2 + c^2b^2 &= 3b^2d^2 \\
        a^2 &= 3b^2 - \frac{b^2 c^2}{d^2}\\
        a^2 &= b^2\left(3 - \frac{c^2}{d^2}\right) \\
        a &= b\sqrt{3 - \frac{c^2}{d^2}}
    \end{align*}
    This shows that $a$ is a multiple of $b$, which contradicts the assumption that $a$ and $b$ are in lowest terms. Therefore the curve $x^2 + y^2 - 3 = 0$ has no rational points.
    \\

    Alternative: Assume towards a contradiction that $x_0 = n_x/d_x, y_0 = n_y/d_y$ all in lowest terms. And let $q = \text{lcm}(d_x, d_y)$. Then there exists $m, n \in \Z$ such that we can rewrite the equation as:
    \begin{align*}
    \frac{m^2}{q^2} + \frac{n^2}{q^2} &= 3\\
    m^2 + n^2 &= 3q^2
    \end{align*}

    with $m, n$, and $q$ in lowest terms.
    \\
    Next we note that any integer squared is either 0 or 1 modulo 4: 0 if it is even, 1 if it is odd. For proof, consider if a number is even it has a factor of 2 and its square has a factor of 4, making it 0 mod 4. If it is odd, then it has the form $2k + 1$ and its square is $4k^2 + 4k + 1 = 4(k^2 + k) + 1$, which is 1 mod 4.\\

    So $m^2$ and $n^2$ are either 0 or 1 mod 4, making the LHS either 0, 1, or 2 mod 4. Likewise $q^2$ is either 0 or 1 mod 4, so $3q^2$ is either 0 or 3 mod 4. The only possible case is that both sides are 0 mod 4. But this means that $m$, $n$, and $q$ were all even, contradicting the assumption that $m$ and $n$, and $q$ were in lowest terms.

\end{solution}

\begin{problem}
    Exercise 20 (above) involved showing that there are no rational points on the curve $x^2 + y^2 - 3 = 0$. Use this fact to show that $\sqrt{3}$ is irrational.
\end{problem}
\begin{solution}{Use proof by contradiction}
    Assume towards a contradiction that $\sqrt{3}$ is rational. Then we could set $x = \sqrt{3}$ and $y = 0$ to get a rational point on the curve $x^2 + y^2 - 3 = 0$. But we know from the previous exercise that there are no rational points on this curve. Therefore $\sqrt{3}$ is irrational.
\end{solution}

\begin{problem}
    Explain why $x^2 + y^2 - 3 = 0$ not having any rational solutions (Exercise 20) implies $x^2 + y^2 - 3^k = 0$ has no rational solutions for $k$ an odd, positive integer.
\end{problem}
\begin{solution}{}
    If $k$ is odd we can write it as $k = 2j + 1$. Then its square root is $\sqrt{3^k} = 3^{\frac{2j +1}{2}} = 3^{j + \frac{1}{2}} = 3^j \sqrt{3}$.
    I claim this number is irrational. Otherwise, we could divide it by $3^j$ (a rational number) to get $\sqrt{3}$ which would also be rational. But if that were the case there would be rational solutions on the curve $x^2 + y^2 - 3 = 0$, which we know is not true.
\end{solution}
\begin{problem}
    Use the above result to prove that $\sqrt{3^k}$ is irrational for all odd, positive $k$.
\end{problem}
\begin{solution}{}
    Fully proven in previous exercise.
\end{solution}

\begin{problem}
    The number $\log_2 3$ is irrational.
\end{problem}
\begin{solution}{Use proof by contradiction}
    Assume towards a contradiction that $\log_2 3$ is rational and let it equal $a/b$. This gives:
    \begin{align*}
        2^{a/b} &= 3 \\
        2^a 2^{1/b} &= 3 \\
        \sqrt[b]{2^a} &= 3\\
        2^a &= 3^b
    \end{align*}
This is a contradiction, since the LHS is even and the RHS is odd.
\end{solution}
\end{document}