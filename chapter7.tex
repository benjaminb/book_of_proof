\documentclass{article}
\usepackage{amsmath} % for advanced math environments
\usepackage{amsfonts} % for math fonts
\usepackage{amssymb} % for math symbols
\usepackage{amsthm} % for theorems and proofs
\usepackage{mathtools} % for mathematical tools
\usepackage{mathrsfs} % for script-like fonts in math
\usepackage{bm} % for bold math symbols
\usepackage{bbm} % for "blackboard-style" characters in math
\usepackage{graphicx} % for including graphics
\usepackage{hyperref} % for including hyperlinks
\usepackage{tcolorbox}
\usepackage{tikz}
\tcbuselibrary{theorems, breakable}
\usepackage{xcolor}
\usepackage[margin=1in]{geometry}

\newcommand{\C}{\mathbb{C}}
\newcommand{\N}{\mathbb{N}}
\newcommand{\Q}{\mathbb{Q}}
\newcommand{\R}{\mathbb{R}}
\newcommand{\Z}{\mathbb{Z}}
\newcommand{\pset}{\mathscr{P}}
\DeclareMathOperator{\lcm}{lcm}

% Define a shortcut for \begin{bmatrix} and \end{bmatrix}
\newcommand{\bmat}[1]{\begin{bmatrix}#1\end{bmatrix}}
\newcommand{\cmat}[1]{\begin{pmatrix}#1\end{pmatrix}}

\newtcolorbox[auto counter]{problem}%
{
    breakable,
    colback=cyan!5,
    colframe=cyan!35!black,
    fonttitle=\bfseries,
    title=Problem~\thetcbcounter,
}

\newtcolorbox{solution}[1]
{
    breakable,
    colback=red!5,
    colframe=red!75!black,
    fonttitle=\bfseries,
    title=Solution: #1,
}

% Title
\title{7 Proving Non-Conditional Statements}
\author{Benjamin Basseri}


\begin{document}

\maketitle

\begin{problem}
Suppose $x \in \Z$. Then $x$ is even if and only if $3x + 5$ is odd.
\end{problem}

\begin{solution}{Use biconditional statements}
  $$x \text{ even} \iff 3x \text{ even} \iff 3x + 5 \text{ odd}$$
\end{solution}

\begin{problem}
Suppose $x \in \Z$. Then $x$ is odd if and only if $3x + 6$ is odd.
\end{problem}
\begin{solution}{Use biconditional statements}
  $$x \text{ odd} \iff 3x \text{ odd} \iff 3x + 6 \text{ odd}$$
\end{solution}

\begin{problem}
Given an integer $a$, then $a^3 + a^2 + a$ is even if and only if $a$ is even.
\end{problem}
\begin{solution}{Prove implication both ways}
  \textbf{Forward direction.} Using the contrapositive argument, we can prove this direction by showing $a$ odd implies $a^3 + a^2 + a$ is odd. Since $a$ is odd, $a^2$ is odd and $a^3$ is odd as well. That means $a^3 + a^2 + a$ is the sum of three odd numbers, which is odd.
  \\
  \textbf{Reverse direction.}. If $a$ is even then $a^2$ is even and so is $a^3$. Then $a^3 + a^2 + a$ is the sum of three even numbers, which is even.
\end{solution}

\begin{problem}
Given an integer $a$, then $a^2 + 4a + 5$ is odd if and only if $a$ is even.
\end{problem}
\begin{solution}{Prove implication both ways}
  \textbf{Forward direction.} Using the contrapositive argument, we can prove this direction by showing $a$ odd implies $a^2 + 4a + 5$ is even. Since $a$ is odd, the expression $a^2 + 4a + 5$ represents (odd) + (odd) + (odd) which is even.
  \\
  \textbf{Reverse direction.}. If $a$ is even then $a^2$ is even and so is $4a$. Then $a^2 + 4a + 5$ is the sum of two even numbers and an odd number, which is odd.
\end{solution}

\begin{problem}
An integer $a$ is odd if and only if $a^3$ is odd.
\end{problem}

\begin{solution}{Prove implication both ways}
  \textbf{Forward direction.} If $a$ is odd then $a^3$ represents (odd)(odd)(odd), which is odd.
  \\
  \textbf{Reverse direction.}. Using the contrapositive argument, if $a$ is even then $a^3$ represents (even)(even)(even), which is even.
\end{solution}

\begin{problem}
Suppose $x, y \in \R$. Then $x^3 + x^2 y = y^2 + xy$ if and only if $y = x^2$ or $y = -x$.
\end{problem}
\begin{solution}{Prove implication both ways}
  \textbf{Forward direction.} If $x^3 + x^2 y = y^2 + xy$ then $x^3 + x^2 y - y^2 - xy = 0$. Factoring, we can rewrite this as $x^2(x + y) = y(x + y)$. If $x + y = 0$ then $y = -x$. Otherwise, $x + y \neq 0$ and we can divide both sides by $x + y$, leaving $x^2 = y$.
  \\
  \textbf{Reverse direction.}. As we saw, if $y = -x$ then both sides simplify to 0. If $y = x^2$ then plugging in $x^2$ for $y$ gives $x^3 + x^4 = x^4 + x^3$, which makes the equation true.
\end{solution}

\begin{problem}
Suppose $x, y \in \R$. Then $(x+y)^2 = x^2 + y^2$ if and only if $x = 0$ or $y = 0$.
\end{problem}
\begin{solution}{Use algebraic manipulation}
  \begin{align*}
    (x+y)^2         & = x^2 + y^2       \\
                    & \Updownarrow      \\
    x^2 + 2xy + y^2 & = x^2 + y^2       \\
                    & \Updownarrow      \\
    2xy             & = 0               \\
                    & \Updownarrow      \\
    xy              & = 0               \\
                    & \Updownarrow      \\
    x = 0           & \text{ or } y = 0
  \end{align*}
\end{solution}

\begin{problem}
Suppose $a, b \in \Z$. Prove that $a\equiv b \pmod{10}$ if and only if $a \equiv b \pmod{2}$ and $a \equiv b \pmod{5}$.
\end{problem}
\begin{solution}{Use definition of modular congruence}
  \textbf{Forward direction.} If $a \equiv b \pmod{10}$ then $10 \mid (a - b)$. This means $2 \mid (a - b)$ and $5 \mid (a - b)$, since 2 and 5 both divide 10. Therefore $a \equiv b \pmod{2}$ and $a \equiv b \pmod{5}$.)
  \\
  \textbf{Reverse direction.}. If $a \equiv b \pmod{2}$ and $a \equiv b \pmod{5}$ then $2 \mid (a - b)$ and $5 \mid (a - b)$. This means $(a-b)$ has a factor of 2 and a factor of 5, so we can write it as $2\cdot5k$ for some integer $k$, or more simply $10k$. Therefore $10 \mid (a - b)$, so $a \equiv b \pmod{10}$.
\end{solution}
\begin{problem}
Suppose $a \in \Z$. Prove that $14 \mid a$ if and only if $7 \mid a$ and $2 \mid a$.
\end{problem}
\begin{solution}{Prove implication both ways}
  \textbf{Forward direction.}
  $$ 14 \mid a \implies 7\cdot 2 \mid a \implies 7 \mid a, 2 \mid a$$
  \\
  \textbf{Reverse direction.}. If $7 \mid a$ and $2 \mid a$ then $a$ has factors of both 7 and 2, and we can write $a = 7\cdot 2k$ for some integer $k$ or more simply $a = 14k$. Therefore $14 \mid a$.
\end{solution}

\begin{problem}
If $a \in \Z$, then $a^3 \equiv a \pmod{3}$.
\end{problem}
\begin{solution}{Use cases}
  Suppose $a \equiv 0 \pmod{3}$. Then $a = 3k$ for some $k$ and $a^3 = 27k^3 = 3(9k^2)$. Therefore 3 divides $a^3$ and $a^3 \equiv a \pmod{3}$.
  \\
  If $a\equiv 1 \pmod{3}$ then we can write $a = 3k + 1$ for some $k$. Its cube is $a^3 = 27k^3 + 27k^2 + 9k + 1 = 3(9k^3 + 9k^2 + 3k) + 1$. Therefore 3 leaves a remainder of 1 after dividing $a^3$, and $a^3 \equiv a \pmod{3}$.
  \\
  If $a\equiv 2 \pmod{3}$ then we can write $a = 3k + 2$ for some $k$. Its cube is $a^3 = 27k^3 + 54k^2 + 36k + 8 = 3(9k^3 + 18k^2 + 12k + 2) + 2$. Therefore 3 leaves a remainder of 2 after dividing $a^3$, and $a^3 \equiv a \pmod{3}$.
  \\
  \textbf{Another approach}: Write $a = 3k + r$ for some remainder $r$. The binomial theorem states
  $$(3k + r)^3 = \sum_{j=0}^3 \binom{3}{j}(3k)^j r^{3-j}.$$
  Notice that every term of the sum above will have a factor of 3 except for when $j = 0$, and that term is just $r^3$. So it suffices to check the cubes of $r$ for $r = 0, 1, 2$. The cubes are 0, 1, and 8, each of which have the same value mod 3 as $r$ itself. Therefore $a^3 \equiv a \pmod{3}$.
\end{solution}

\begin{problem}
Suppose $a,b \in \Z$. Prove that $(a-3)b^2$ is even if and only if $a$ is odd or $b$ is even.
\end{problem}
\begin{solution}{Prove implication both ways}
  \textbf{Forward direction.} Prove the contrapositive statement: if $a$ is even and $b$ is odd, then $(a-3)$ is odd and $b^2$ is odd. This makes $(a-3)b^2$ an odd times an odd, which is odd.
  \textbf{Reverse direction.} If $a$ is odd then $a - 3$ is even, making the product $(a-3)b^2$ even. If $b$ is even then $b^2$ is even, making the product $(a-3)b^2$ even.
\end{solution}

\begin{problem}
There exists a positive real number $x$ for which $x^2 < \sqrt{x}$.
\end{problem}
\begin{solution}{}
  Positive reals get bigger when you square them if they're above 1, and they get smaller if they're between 0 and 1. So consider a number $k > 1$,
  $$\left(\frac{1}{k}\right)^4 < \left(\frac{1}{k}\right)^2 < \frac{1}{k}$$
  Notice that $1/k^2$ has the property that it's a positive real number and its square is less than its root.
\end{solution}
\begin{problem}
Suppose $a, b \in \Z$. If $a+ b$ is odd, then $a^2 + b^2$ is odd.
\end{problem}
\begin{solution}{Use direct proof}
  Suppose $a + b$ is odd. Then one of the two terms must be odd and the other is even, making it (odd) + (even). Squaring preserves parity, so $a^2 + b^2$ reduces to (odd) + (event), which is odd.
\end{solution}

\begin{problem}
Suppose $a \in \Z$. Then $a^2 \mid a$ if and only if $a \in \{-1, 0, 1\}$.
\end{problem}
\begin{solution}{Prove implication both ways}
  \textbf{Forward direction.} Using the contrapositive statement, assume that $a \not\in \{-1, 0, 1\}$. Then $a$ is at least 2 or less than -2. For any such integer its square is strictly bigger than the base, so $a^2$ could not be a factor of $a$.
  \\

  \textbf{Reverse direction.} If $a \in \{-1, 0, 1\}$ then $a^2$ is 1, 0, or 1 respectively, and each of these divides $a$.

\end{solution}

\begin{problem}
Suppose $a,b\in\Z$. Prove that $a+b$ is even if and only if $a$ and $b$ have the same parity.
\end{problem}
\begin{solution}{Prove implication both ways}
  \textbf{Forward direction.} If $a + b$ is even then either $a$ and $b$ are both even or both odd. If they had different parity, the expression would simplify to (odd)  + (even), which is odd.
  \\
  \textbf{Reverse direction.} The sum of two evens is even and the sum of two odds is even. So if $a$ and $b$ have the same parity then $a + b$ is even.
\end{solution}

\begin{problem}
Suppose $a,b\in\Z$. If $ab$ is odd, then $a^2 + b^2$ is even.
\end{problem}
\begin{solution}{Use direct proof}
  If $ab$ is odd, it must be that both $a$ and $b$ are odd. Otherwise, $ab$ would be the product of an even number and another number, which would be even. Squaring preserves parity, so $a^2$ and $b^2$ are both odd. This makes $a^2 + b^2$ the sum of two odds, which is even.
\end{solution}

\begin{problem}
There is a prime number between 90 and 100.
\end{problem}
\begin{solution}{Show the example}
  The prime number 97 is between 90 and 100.
\end{solution}

\begin{problem}
There is a set $X$ for which $\N \in X$ and $\N \subseteq X$.
\end{problem}
\begin{solution}{Construct an example}
  Let $X = \N \cup \{\N\}$. Then $\N$ is both a member of and a subset of $X$.
\end{solution}

\begin{problem}
If $n\in\N$, then $2^0 + 2^1 + 2^2 + \cdots + 2^n = 2^{n+1} - 1$.
\end{problem}
\begin{solution}{Use induction}
  \textbf{Base case.} Let $n = 1$. Then we have $2^0 + 2^1 = 3$ and  $2^{n+1} - 1 = 2^2 - 1 = 3$ as well.
  \\

  \textbf{Inductive hypothesis:} Assume that $2^0 + 2^1 + 2^2 + \cdots + 2^n = 2^{n+1} - 1$.
  \\

  \textbf{Inductive step:} Consider the $n+1$ case. The sum becomes:
  $$\sum_{k=0}^{n+1} 2^k = 2^{n+1} + \sum_{k=0}^{n} 2^k$$

  By the inductive hypothesis, $\sum\limits_{k=0}^{n} 2^k = 2^{n+1} - 1$. Therefore the entire sum becomes:
  $$2^{n+1} + 2^{n+1} - 1 = 2\cdot 2^{n+1} - 1 = 2^{n+2} - 1$$

  which confirms the inductive step.

\end{solution}

\begin{problem}
There exists an $n \in \N$ for which $11 \mid (2^n - 1)$.
\end{problem}
\begin{solution}{Show an example}
  We want to find a number such that is one more than a multiple of 11. The number 10 works, since $2^{10} = 1024$ and $1024 - 1 = 1023 = 11\cdot 93$. Therefore $11 \mid (2^{10} - 1)$.
\end{solution}

\begin{problem}
Every real solution of $x^3 +x + 3 = 0$ is irrational.
\end{problem}
\begin{solution}{Use contradiction}
  Suppose there is a real solution $x = \frac{p}{q}$ where $p$ and $q$ are integers in lowest terms. Then we can rewrite the equation as:
  $$\left(\frac{p}{q}\right)^3 + \frac{p}{q} + 3 = 0$$
  Multiplying by $q^3$ gives:
  $$p^3 + pq^2 + 3q^3 = 0$$
  $$\implies p^3 = q^2(-p - 3q)$$
  This means $p^3$ is a multiple of $q$, which implies $p$ is a multiple of $q$. But this contradicts the assumption that $p/q$ was in lowest terms.
\end{solution}

\begin{problem}
If $ n \in \Z$ then $4 \mid n^2$ or $4 \mid (n^2 - 1)$.
\end{problem}
\begin{solution}{Use cases}
  Split into two cases: $n$ is even or $n$ is odd.
  \\

  If $n$ is even, then $n = 2k$ for some integer $k$. Then $n^2 = 4k^2$ which is divisible by 4.
  \\

  If $n$ is odd then $n = 2k + 1$ for some integer $k$, and its square is $4k^2 + 4k + 1 = 4(k^2 + k) + 1$, which has a remainder of 1 when divided by 4. Therefore $4 \mid (n^2 - 1)$.
\end{solution}

\begin{problem}
Suppose $a, b$ and $c$ are integers. If $a \mid b$ and $a \mid (b^2 - c)$, then $a \mid c$.
\end{problem}
\begin{solution}{Represent with modular forms}
  Since $a \mid b$ as $b \equiv 0 \pmod{a}$. Then $b^2 \equiv 0 \pmod{a}$ as well. And since $a \mid (b^2 - c)$, this means $b^2 \equiv c \pmod{a}$. Since $b^2$ is equivalent to 0 and $c$ modulo $a$ it must be that $c \equiv 0 \pmod{a}$ (modular congruence is transitive), which means that $c$ is a multiple of $a$ and $a \mid c$.
\end{solution}

\begin{problem}
If $a \in \Z$, then $4 \nmid (a^2 - 3)$.
\end{problem}
\begin{solution}{Use key fact that squares are 0 or 1 mod 4}
  Note that any integer $a$, when squared, is 0 or 1 mod 4. Therefore $a^2 - 3$ is either 1 or 2 mod 4 which means indivisible by 4.
\end{solution}

\begin{problem}
If $p > 1$ is an integer and $n \nmid p$ for each integer $n$ for which $2 \leq n \leq \sqrt{p}$, then $p$ is prime.
\end{problem}
\begin{solution}{Use contrapositive}
  The contrapositive statement is if integer $p > 1$ is composite, there exists an $n$ between 2 and $\sqrt{p}$ that divides $p$.
  \\

  Given $p$ is composite, it has some factor $n$. Suppose $n = \sqrt{p}$, then $n^2 = p$ and $n \mid p$. Otherwise, $n$ is either strictly greater or stricly less than $\sqrt{p}$.
  \\

  If $n < \sqrt{p}$ then we have satistied the proof. But if $n > \sqrt{p}$ then it must have a corresponding factor $m$ such that $nm = p$ and $m < \sqrt{p}$. For if $m > \sqrt{p}$, the product $nm$ would exceed $p$ as both factors would be larger than $\sqrt{p}$. Therefore $m < \sqrt{p}$ and $m$ satisfies the statement.
\end{solution}

\begin{problem}
The product of any $n$ consecutive positive integers is divisible by $n!$.
\end{problem}
\begin{solution}{Use induction}
  \textbf{Base case.} Let $n = 1$. Then the product of 1 consecutive positive integers is 1, which is divisible by 1!.
  \\

  \textbf{Inductive hypothesis.} Assume that the product of $n$ consecutive positive integers is divisible by $n!$.
  \\

  \textbf{Inductive step.} Take the $n+1$ case, and its product $k(k+1)\ldots(k+n+1)$. If the inductive hypothesis product $k\ldots(k+n)$ was divisible by $(n+1)!$ then so is the inductive step product $k\ldots(k+n+1)$ and the proof is done. Otherwise, the hypothesis product does not have a factor of $n+1$. Moving onto the step's product $k\ldots(k+n+1)$, by the pigeonhole principle it must contain a multiple of $n+1$ whereas the previous product did not. Therefore we can say $k\ldots(k+n+1)$ is divisible by $n!$ and has a 'new' multiple of $n+1$ which makes it divisible by $(n+1)!$.
\end{solution}

\begin{problem}
Suppose $a, b \in Z$ If $a^2 + b^2$ is a perfect square, then $a$ and $b$ are not both odd.
\end{problem}
\begin{solution}{Use key fact that squares are 0 or 1 mod 4 to derive a contradiction}
  If $a^2 + b^2$ is a perfect square then it is 0 or 1 mod 4. If $a$ and $b$ were both odd their squares would be 1 mod 4, and their sum would be 2 mod 4, which is a contradiction.
\end{solution}

\begin{problem}
Prove the division algorithm: If $a, b \in \N$, there exist \textit{unique} integers $q, r$ for which $a = bq + r$ and $0 \leq r < b$.
\end{problem}
\begin{solution}{Use mutual inequality to prove uniqueness}
  The existence was proven in the text. We know that the coefficient on $b$ is the largest non-negative multiple of $b$ that does not exceed $a$, so to prove $q$'s uniqueness we need to show that this maximum is unique.

  Let $M = \{q \in \Z: 0 \leq bq \leq a \}$, the set of non-negative multiples of $b$ that do not exceed $a$. Since $0 \in M$, $M$ is non-empty. To show its maximum is unique, suppose $q_1$ and $q_2$ both have the property that for all $x \in M$, $x \leq q_1$ and $x \leq q_2$. Then $q_1 \leq q_2$ and $q_2 \leq q_1$, which means $q_1 = q_2$. Therefore the coefficient on $b$ is unique.

  Going back to the algorithm $a = qb + r$ we can solve for $r$ by writing $a - qb = r$. This has a unique solution in the integers, therefore $r$ is unique.
\end{solution}

\begin{problem}
If $a \mid bc$ and $\gcd(a, b) = 1$, then $a \mid c$.
\end{problem}
\begin{solution}{}
  Suppose $\gcd(a,b) = 1$. If $a = 1$ then $a$ divides $c$. Otherwise, $a$ and $b$ have no factors in common, including $a$ itself. Therefore $a$ cannot divide $b$. But since $a$ divides the product $bc$, it must divide $c$.
\end{solution}

\begin{problem}
Suppose $a, b, p \in \Z$ and $p$ is prime. If $p \mid ab$ then $p \mid a$ or $p \mid b$.
\end{problem}
\begin{solution}{Use prime decomposition}
  Since $p$ divides the product $ab$, it must be that $p$ appears in the prime decomposition of $a$, $b$ or both. Since $p$ is prime, it cannot be that it is the product of some prime in $a$ and another in $b$, making it a factor of neither. Therefore $p$ must divide $a$ or $b$.
\end{solution}

\begin{problem}
If $n \in \Z$, then $\gcd(n, n+1) = 1$.
\end{problem}
\begin{solution}{{Direct proof, rewriting $n, n+1$ as multiples of $d$}}
  Let $d = \gcd(n, n+1)$. Then $n = dx$ and $n + 1 = dy$ for some integers, $x, y$. Then write:
  $$n + 1 - n = 1 = dy - dx = d(y - x)$$
  Since $d, x, y$ are all integers, for $d(y - x)$ to equal 1 $d$ must be $\pm1$. And since $1$ is the greatest of the two and indeed a valid divisor for any $n, n+1$, we have $\gcd(n, n+1) = 1$.
\end{solution}
\begin{problem}
If $n \in \Z$ then $\gcd(n, n+2) \in \{1, 2\}$.
\end{problem}
\begin{solution}{{Use direct proof, rewriting $n, n+2$ as multiples of $d$}}
  Let $d = \gcd(n, n+2)$. Then $n = dx$ and $n + 2 = dy$ for some integers, \(x, y\). Then write:
  \[n + 2 - n = 2 = dy - dx = d(y - x)\]
  Since \(d\) is a positive integer it must be either 1 or 2.
\end{solution}

\begin{problem}
If \(n \in \Z\), then \(\gcd(2n+1, 4n^2+1) = 1\).
\end{problem}
\begin{solution}{Express one of the numbers in terms of the other}
  Let \(d = \gcd(2n+1, 4n^2+1)\). Then \(2n+1 = dx\) and \(4n^2+1 = dy\) for some integers \(x, y\). However, we can rewrite \(4n^2 + 1 = (2n+1)(2n-1) +2\). Using \(dx\) we can then say:
  \begin{align*}
    (2n+1)(2n-1) + 2 & = dy             \\
    dx(2n-1) + 2     & = dy             \\
    2                & = dy - dx(2n-1)  \\
    2                & = d(y - x(2n-1))
  \end{align*}
  This shows that \(d\) divides 2, so it must be 1 or 2. However we also know \(dx = 2n + 1\), an odd number. Since only odd numbers can multiply to an odd number, \(d\) must be odd, which means \(d = 1\).
\end{solution}

\begin{problem}
Suppose \(a, b \in \N\). Then \(a = \gcd(a, b)\) if and only if \(a \mid b\).
\end{problem}
\begin{solution}{Prove implication both ways}
  \textbf{Forward direction}. Suppose \(a = \gcd(a, b)\). Then \(a\) divides \(b\) so \(a \mid b\).
  \\

  \textbf{Reverse direction}. Suppose \(a \mid b\). Of course \(a\) divides itself so \(a\) divides both \(a\) and \(b\) making it a common divisor. No number higher than \(a\) can divide \(a\) so it is the greatest common divisor.
\end{solution}
\begin{problem}
Suppose \(a, b \in \N\). Then \(a = \lcm(a, b)\) If and only if \(b \mid a\).
\end{problem}
\begin{solution}{Prove implication both ways}
  \textbf{Forward direction}. Suppose \(a = \lcm(a, b)\). Then \(a\) is a multiple of \(b\) so \(b \mid a\).
  \\

  \textbf{Reverse direction}. Suppose \(b \mid a\), making \(b\) a multiple of \(a\). It is also trivially a multiple of itself, so it is a common multiple. No number smaller than \(b\) can be a multiple of \(b\), so \(b\) is the least common multiple.
\end{solution}
\end{document}
