\documentclass{article}
\usepackage{amsmath} % for advanced math environments
\usepackage{amsfonts} % for math fonts
\usepackage{amssymb} % for math symbols
\usepackage{amsthm} % for theorems and proofs
\usepackage{mathtools} % for mathematical tools
\usepackage{mathrsfs} % for script-like fonts in math
\usepackage{bm} % for bold math symbols
\usepackage{bbm} % for "blackboard-style" characters in math
\usepackage{graphicx} % for including graphics
\usepackage{hyperref} % for including hyperlinks
\usepackage{tcolorbox}
\usepackage{tikz}
\tcbuselibrary{theorems, breakable}
\usepackage{xcolor}
\usepackage[margin=1in]{geometry}

\newcommand{\C}{\mathbb{C}}
\newcommand{\N}{\mathbb{N}}
\newcommand{\Q}{\mathbb{Q}}
\newcommand{\R}{\mathbb{R}}
\newcommand{\Z}{\mathbb{Z}}
\newcommand{\pset}{\mathscr{P}}
\DeclareMathOperator{\lcm}{lcm}

% Define a shortcut for \begin{bmatrix} and \end{bmatrix}
\newcommand{\bmat}[1]{\begin{bmatrix}#1\end{bmatrix}}
\newcommand{\cmat}[1]{\begin{pmatrix}#1\end{pmatrix}}

\newtcolorbox[auto counter]{problem}%
{
    breakable,
    colback=cyan!5,
    colframe=cyan!35!black,
    fonttitle=\bfseries,
    title=Problem~\thetcbcounter,
}

\newtcolorbox{solution}[1]
{
    breakable,
    colback=red!5,
    colframe=red!75!black,
    fonttitle=\bfseries,
    title=Solution: #1,
}

% Title
\title{Your Document Title}
\author{Your Name}
\date{\today} % or specify a date like {December 2020}

\begin{document}

\maketitle
\begin{problem}
Prove that $\lim\limits_{x\to 5} (8x - 3) = 37$.
\end{problem}

For scratch work:
\begin{align*}
  |8x - 3 - 37| & < \epsilon           \\
  |8x -40 |     & < \epsilon           \\
  8|x - 5|      & < \epsilon           \\
  |x - 5|       & < \frac{\epsilon}{8}
\end{align*}

Then for any $\epsilon > 0$ let $\delta < \frac{\epsilon}{8}$. Then if $|x - 5 | < \delta < \epsilon/8$ we have $8|x - 5| < \epsilon$ or $|8x - 3 - 37| < \epsilon$.

\begin{problem}
Prove that $\lim\limits_{x \to -1} (4x + 6) = 2$.
\end{problem}


For scratch work:
\begin{align*}
  |4x + 6 - 2| & < \epsilon           \\
  |4x + 4|     & < \epsilon           \\
  4|x + 1|     & < \epsilon           \\
  4|x - (-1)|  & < \epsilon           \\
  |x - (-1)|   & < \frac{\epsilon}{4}
\end{align*}

Then for any $\epsilon > 0$ use $\delta < \epsilon / 4$. If $|x - (-1)| < \delta < \epsilon / 4$ then $4|x + 1| < \epsilon$ or $|4x + 6 - 2| < \epsilon$.

\begin{problem}
Prove that $\lim\limits_{x\to 0}(x + 2) = 2$.
\end{problem}

For scratch work:
\begin{align*}
  |x + 2 - 2| & < \epsilon \\
  |x|         & < \epsilon \\
\end{align*}

So for any $\epsilon > 0$ use $|x| < \epsilon$. Then $|x + 2 - 2| < \epsilon$

\begin{problem}
Prove that $\lim\limits_{x \to 8} (2x - 7) = 9$.
\end{problem}

Scratch work:

\begin{align*}
  |2x - 7 - 9| & < \epsilon           \\
  |2x - 16|    & < \epsilon           \\
  2|x - 8|     & < \epsilon           \\
  |x - 8|      & < \frac{\epsilon}{2} \\
\end{align*}

So for any $\epsilon > 0$ use $\delta < \frac{\epsilon}{2}$. Then $|x - 8| < \delta$ implies $|x - 8| < \frac{\epsilon}{2}$ and $2|x - 8| = |2x - 7 -9| < \epsilon$.

\begin{problem}
Prove that $\lim\limits_{x\to 3} (x^2 - 2) = 7$.
\end{problem}

Scratch work:
\begin{align*}
  |x^2 - 2 - 7|  & < \epsilon \\
  |x^2 - 9|      & < \epsilon \\
  |x - 3||x + 3| & < \epsilon \\
  |x - 3||x + 3| & < \epsilon \\
\end{align*}

Now as $x$ approaches 3, the $x + 3$ term approaches 6. We can constrain $|x - 3| < 1$ so that $x$ is within 1 unit of 3. Since $x$ is bounded above by 4, $x + 3$ must be less than 7. So we can continue the scratch work:
\begin{align*}
  |x - 3||x + 3| & < |x - 3|7           & < \epsilon \\
  |x - 3|        & < \frac{\epsilon}{7}
\end{align*}

So for any $\epsilon > 0$ use $\delta < \min\{1, \epsilon/7\}$. Then

$$|x - 3| < \delta \implies |x - 3| < \min\{1, \epsilon/7\}$$

If $|x - 3| < \delta$ then $x$ is within 1 unit of 3 so $|x + 3| < 7$ and $7|x - 3| < \epsilon$, so $|x + 3||x - 3| < \epsilon$, or that $|x^ - 2 - 7| < \epsilon$.

\begin{problem}
Prove that $\lim\limits_{x \to 1} (4x^2 + 1) = 5$.
\end{problem}

Scratch work:

\begin{align*}
  |4x^2 + 1 - 5| & < \epsilon           \\
  4|x^2 - 1|     & < \epsilon           \\
  |x - 1||x + 1| & < \frac{\epsilon}{4}
\end{align*}

As $x$ approaches 1 $x + 1$ approaches 2, so we can force $x$ to be close enough to 1 that $x + 1$ is bounded above by 3, making
$$|x - 1||x + 1| < 3|x - 1| < \frac{\epsilon}{4}$$
$$\implies |x - 1| < \frac{\epsilon}{12}$$

Then for any $\epsilon > 0$ choose $\delta < \min\{1, \epsilon/12\}$. If $|x - 1| < \delta$ then $x$ is within 1 unit of 1, making $|x + 1| < 3$ and $3|x - 1| < \epsilon/4$. Then $|x + 1||x - 1| < \epsilon$ and $|4x^2 + 1 - 5| < \epsilon$.

\end{document}