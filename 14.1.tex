\documentclass{article}
\usepackage{amsmath} % for advanced math environments
\usepackage{amsfonts} % for math fonts
\usepackage{amssymb} % for math symbols
\usepackage{amsthm} % for theorems and proofs
\usepackage{mathtools} % for mathematical tools
\usepackage{mathrsfs} % for script-like fonts in math
\usepackage{bm} % for bold math symbols
\usepackage{bbm} % for "blackboard-style" characters in math
\usepackage{graphicx} % for including graphics
\usepackage{hyperref} % for including hyperlinks
\usepackage{tcolorbox}
\usepackage{tikz}
\tcbuselibrary{theorems, breakable}
\usepackage{xcolor}
\usepackage[margin=1in]{geometry}

\newcommand{\C}{\mathbb{C}}
\newcommand{\N}{\mathbb{N}}
\newcommand{\Q}{\mathbb{Q}}
\newcommand{\R}{\mathbb{R}}
\newcommand{\Z}{\mathbb{Z}}
\newcommand{\pset}{\mathscr{P}}
\DeclareMathOperator{\lcm}{lcm}

% Define a shortcut for \begin{bmatrix} and \end{bmatrix}
\newcommand{\bmat}[1]{\begin{bmatrix}#1\end{bmatrix}}
\newcommand{\cmat}[1]{\begin{pmatrix}#1\end{pmatrix}}

\newtcolorbox[auto counter]{problem}%
{
    breakable,
    colback=cyan!5,
    colframe=cyan!35!black,
    fonttitle=\bfseries,
    title=Problem~\thetcbcounter,
}

\newtcolorbox{solution}[1]
{
    breakable,
    colback=red!5,
    colframe=red!75!black,
    fonttitle=\bfseries,
    title=Solution: #1,
}

% Title
\title{Your Document Title}
\author{Benjamin Basseri}


\begin{document}

\maketitle

\textbf{A.} Show that the two given sets have equal cardinality by describing a bijection from one to the other. Describe your bijection with a formula (not as a table).

\begin{problem}
$\R$ and $(0, \infty)$
\end{problem}

Look for a function that takes any real number and outputs any positive real number. Many such functions exist, such as $f(x) = e^x$. As $x$ gets very negative, $e^x$ approaches 0 and $e^x$ can grow without bound as $x$ increases. And since $e^x$ is invertible on the positive reals (using log), this is a bijection.

\begin{problem}
$\R$ and $(\sqrt{2}, \infty)$
\end{problem}

The target set $(\sqrt{2}, \infty)$ only differs from $(0, \infty)$ by the lower endpoint. Take the function $f(x) = e^x$ and follow it with the function $g(x) = x + \sqrt{2}$. Put together:
$$h(x) = g(f(x)) = e^x + \sqrt{2}$$

The function $h$ is bijective since it's the composition of bijective functions: $e^x$ on a positive set and $g(x)$ inverse would simply be to subtract $\sqrt{2}$.

\begin{problem}
$\R$ and $(0, 1)$
\end{problem}

In the text they prove this by showing transitively $|\R| = |(0, \infty)| = |(0, 1)|$. We could compose the two functions used to get a bijection:
$$h: \R \longrightarrow (0, 1)$$
$$x \longmapsto \frac{2^x}{2^x + 1}$$

Taking $f(x) = \frac{x}{x + 1}$ (a bijection between $(0, \infty)$ and $(0, 1)$) and $g(x) = 2^x$ (a bijection between $\R$ and $(0, \infty)$), we compose them to make $h = g \circ f$. As a composition of bijections, $h$ is bijective as well.

\begin{problem}
The set of even integers and the set of odd integers.
\end{problem}

Let $f(x) = x + 1$ mapping even integers to odd integers. Its inverse function is the `minus 1' function, therefore $f$ is a bijection between evens and odds.

\begin{problem}
$A = \{3k: k \in \Z\}$ and $B = \{7k : k \in \Z\}$.
\end{problem}

Both sets $A$ and $B$ are in a sense generated from $\Z$. In fact, if we divide each $a \in A$ by 3 we'll return the the set of integers $\Z$, which we can then multiply by $7$ to convert to the set $B$. So we may suspect that the map $f(x) = \frac{7}{3}x$ will be our bijection. It maps $A$ members to $B$ members as just described, and $f$ is invertible by multiplying by $3/7$. Therefore $f$ is a bijection between $A$ adn $B$ and $|A| = |B|$.

\begin{problem}
$\N$ and $S = \{\frac{\sqrt{2}}{n}: n \in \N\}$
\end{problem}

Similar to the previous problem, we see that $S$ is generated by $\N$ and in fact to transform $n \in N$ to an $s \in S$ we take $n \mapsto \sqrt{2}/n$. This map is surjective because this is how all elements in $S$ are specified. It is also injective since if $n \neq m$ then $1/n \neq 1/m$ and $\sqrt{2}/n \neq \sqrt{2}/m$. Therefore the function $f(n) = \sqrt{2}{n}$ is a bijection between these sets and $|\N| = |S|$.

\begin{problem}
$\Z$ and $S = \{\ldots, \frac{1}{8}, \frac{1}{4}, \frac{1}{2}, 1, 2, 4, 8, 16, \ldots\}$
\end{problem}

The set $S$ seems to be all integer powers of 2. So the simplest map from $\Z$ to $S$ would be $f(x) = 2^x$. This is invertible by the $\log_2$ function, so a bijection exists and the sets have equal cardinality.

\begin{problem}
$\Z$ and $S = \{x \in \R: \sin x = 1\}$
\end{problem}

We know that $\sin x = 1$ when $x = \pi/2$ or a multiple of $2\pi$ plus $\pi/2$:
$$\sin x = 1 \implies x = \frac{\pi}{2} + 2\pi k, k \in \Z$$

So we could rewrite $S$ as
$$S = \{x \in \R: x = \pi/2 + 2\pi k, k \in \Z\}$$

So the function $f(k) = \pi/2 + 2\pi k$ maps integers to $S$ elements. This function is also invertible:
$$f^{-1}(y) = \frac{y - \pi/2}{2\pi}$$

So the bijection exists between $\Z$ and $S$ and $|\Z| = |S|$.

\begin{problem}
$\{0, 1\} \times \N$ and $\N$
\end{problem}

Similar to the table that shows $|\N| = |\Z|$ we can write a table:

$$
  \begin{array}{c|cccccc}
    \N                 & 1      & 2      & 3      & 4      & 5      & \cdots \\
    \hline
    \{0, 1\} \times \N & (0, 1) & (1, 1) & (0, 2) & (1, 2) & (0, 3) & \cdots \\
  \end{array}
$$

however the problem asks us to specify an actual formula. We do see that when $n \in \N$ is odd the first coordinate in $\{0, 1\} \times \N$ is 0, and it's 1 otherwise. The second coordinate is $(n + 1) // 2$ (floor division). Using these coordinate functions we can compose a bijection:
$$f(n): \N \longrightarrow \{0, 1\} \times \N$$
$$n \longmapsto (n + 1 \pmod{2}, (n + 1) // 2)$$

To show that $f$ is a bijection, we'll construct the inverse function.
$$g(a, b) = 2b - (1 - a)$$

If $a = 0$ then $n$ was odd, and the result of $(n + 1) / 2 = b$ since we can drop the floor division and use regular division for odd $n$. Solving for $n$ gives us $n = 2b - 1$, which matches the formula when $a = 0$.

If $a = 1$ then $n$ was even and $n / 2 = (n + 1) // 2 = b$. Solving for $n$ we have $n = 2b$, which matches our formula as well for when $a = 1$.

\begin{problem}
$\{0, 1\} \times \N$ and $\Z$
\end{problem}

If we use $\{0, 1\}$ to encode the sign we'd have an easy time of it:
$$
  \begin{array}{c|cccccc}
    \{0, 1\} \times \N & (0, 1) & (1, 1) & (0, 2) & (1, 2) & \cdots \\
    \hline
    \N                 & -1     & 1      & -2     & 2      & \ldots
  \end{array}
$$

However we have to include 0 in the mapping, so try shifting everything to the right:
$$
  \begin{array}{c|ccccccc}
    \{0, 1\} \times \N & (0, 1) & (1, 1) & (0, 2) & (1, 2) & (0, 3) & \cdots \\
    \hline
    \N                 & 0      & -1     & 1      & -2     & 2      & \ldots
  \end{array}
$$

Call the tuple $(a, b)$. We can encode the sign as $(-1)^a$ so when $a = 1$ we have a -1 coefficient, and it simplifies to 1 when $a = 0$. For the magnitude, we see it's equal to $b$ when $a$ is 1 or `switched on', and it's $b - 1$ when $a$ is `off'. So we can use a one-hot switch to encode that: $b - (1 - a)$. Putting this together we have
$$f(a, b) = (-1)^a(b - (1 - a))$$

Prove this is a bijection by showing the inverse function. First, if $x \in Z = 0$ then $f^{-1}(0) = (0, 1)$.

For any other $x \in \Z$, we used $a$ to encode its sign:
$$a = \begin{cases}
    1, & x < 0 \\
    0, & x > 0
  \end{cases}
$$

And we used $b$ to encode its magnitude:
$$
  b = \begin{cases}
    |x|,     & x < 0 \\
    |x| + 1, & x > 0
  \end{cases}
$$

Put together we have:
$$f^{-1}(x) = \begin{cases}
    (1, |x|),     & x < 0 \\
    (0, |x| + 1), & x > 0
  \end{cases}$$

To show this is indeed the inverse, suppose $x = (-1)^a (b - (1 - a))$. If $x > 0$ then it must be $a = 0$ and $|x| = |b - 1|$. However since $x > 0$ and $b - 1 \geq 0$ we can drop the absolute value bars and state $x = b - 1$ or $x + 1 = b$. So we have $(a, b) = (0, x + 1)$ which equals $(0, |x| + 1)$, matching $f^{-1}$. If $x < 0$ then $a = 1$ and $|x| = |b|$. Since $b > 0$ we have $|b| = b$ and more simply, $(a, b) = (1, |x|)$, again as $f^{-1}$ predicts. Therefore $f^{-1}$ in all cases takes an integer $x$ and returns the $(a, b)$ pair the mapped to it on $f$.

\begin{problem}
$[0, 1]$ and $(0, 1)$.
\end{problem}

Here take inspiration from Hammack's proof but note there is a mistake in his solution: his function $g: [0, 1) \to (0, 1)$ actually has codomain $[0, 1)$ since as defined, $g(0) = 0$. Thus it doesn't achieve the goal of mapping into the open interval.

The trick here is to take the unwanted endpoints, 0 and 1, and map them into the interval's interior in a way that won't overlap with other mappings. For instance, we can get rid of 1 by sending the natural reciprocals, $\frac{1}{n}$ for $n \in \N$ to $\frac{1}{n+1}$ and leaving everything else alone.

Formally: let $X = \{1/n: n \in \N\}$ and define $f$:
$$f(x) = \begin{cases}
    x             & x \in [0, 1] - X \\
    \frac{1}{x+1} & \text{otherwise}
  \end{cases}
$$

We see that $f(1) = 1/2$ and anything less than $1$ must map into $[0, 1)$. The function $f$ is invertible: on the first branch it's the identity function so nothing happened to the input, and on the second branch we mapped a natural reciprocal to the next natural reciprocal (e.g. $1/3 \mapsto 1/4 \mapsto 1/5 \mapsto \ldots)$ so we just have to map it back:
$$f^{-1}(x) = \begin{cases}
    x               & x \in [0, 1] - X \\
    \frac{1}{x - 1} & x \in X
  \end{cases}
$$

So we have a bijection between $[0, 1]$ and $[0, 1)$. We can do something similar to shave off the left endpoint 0. Before, we sent $1 \mapsto 1/2 \mapsto 1/3 \mapsto 1/4 \mapsto \ldots$. Now we need a similar chain but starting with 0. So if we want $0 \mapsto 1/2 \mapsto 1/3 \mapsto 1/4 \mapsto \ldots$ we can define the branch $n \mapsto \frac{1}{n + 2}$ for $n \in \N \cup \{0\}$. So let $Y = \{\frac{1}{n+2}: n \in \N \cup \{0\}\}$ and our second function is
$$g(x) = \begin{cases}
    x ,              & x \in [0, 1) - Y \\
    \frac{1}{n + 2}, & x \in Y
  \end{cases}$$

The $g$ function is also bijective since any natural reciprocal follows from second branch, and $\frac{1}{x} \mapsto \frac{1}{x - 1}$ unless $x = 2$, then $g^{-1}(1/2) = 0$. Otherwise, $g(x)$ is on the identity branch and $g^{-1}$ is the identity. Now we have a bijective function $g: [0, 1) \to (0, 1)$.

And we can have $g \circ f$ which bijectively maps from $[0, 1]$ to $(0, 1)$, proving they have equal cardinality.

\begin{problem}
$\N$ and $\Z$
\end{problem}

As proven in section 12.2, one such bijection is
$$f: \N \longrightarrow \Z$$
$$n \longmapsto \frac{(-1)^n(2n - 1) + 1}{4}$$

\begin{problem}
$\mathcal{P}(\N)$ and $\mathcal{P}(\Z)$
\end{problem}

Since $\N$ and $\Z$ have the same cardinality it makes sense that their power sets have the same cardinality as well. Since $|\N| = |\Z|$ there exists a bijection $f: \N \to \Z$. To define a bijection between their powersets we'll use $f$ again. Define
$$F: \mathcal{P}(\N) \to \mathcal{P}(\Z)$$
$$ A \subseteq \N \longmapsto B = \{f(a): a \in A\}$$

So $F$ is basically the set-valued version of $f$. In any $A \subseteq \N$, the $f$ function uniquely maps the members to integers encoding a subset $B$ in $\Z$. Since $f$ is bijective we could take the integers in $B$ and map them back to $A$ on $f^{-1}$.

\begin{problem}
$\N \times \N$ and $\{(n, m) \in \N \times \N: n \leq m\}$
\end{problem}

Although we need a formula, not a table, we can use a table for the second set for inspiration.

$$\begin{array}{c|cccc}
    n \backslash m & 1     & 2     & 3     & 4     \\
    \hline
    1              & (1,1) & (1,2) & (1,3) & (1,4) \\
    2              &       & (2,2) & (2,3) & (2,4) \\
    3              &       &       & (3,3) & (3,4) \\
    4              &       &       &       & (4,4) \\
  \end{array}$$

Notice that the table for the second set is the upper triangle of the table for $\N \times \N$. One bijection might be starting with $\N \times \N$ and shifting the pairs to the right, since this seems to happen in a simple pattern. So the map would be somthing like $f(n, m) = (n, m + x)$. We can see that the second coordinate controls how far to the right something gets shifted, it is second coordinate minus one. The bijection then is
$$f(n, m) = (n, n + m - 1)$$

To show that $f$ is a bijection we can show its inverse: on the first coordinate $f$ is the identity, on the second coordinate we have $m \mapsto n + m - 1$. Therefore for any $(p, q)$ in the second set we can undo $f$ by subtracting the first coordinate $p$ and adding back 1:
$$f^{-1}(p, q) = (p, q - p + 1)$$

\textbf{B.} Answer the following questions concerning bijections from this section.

\begin{problem}
Find a formula for the bijection $f$ in Example 14.2, $|\N| = |\Z|$
\end{problem}

The table given looks like:
$$
  \begin{array}{c|ccccccc}
    n    & 1 & 2 & 3  & 4 & 5  & 6 & 7  \\
    \hline
    f(n) & 0 & 1 & -1 & 2 & -2 & 3 & -3 \\
  \end{array}
$$

Apart from $f(1) = 0$ the parity of $n \in \N$ determines the sign of $m \in \Z$ with even naturals mapping to positive integers and odd naturals mappping to negative integers. We also see that if $n$ is even it maps to $n/2$, but if $n$ is odd it maps to the negative of $(n - 1) / 2$. So the function can be written as:
$$f(n) = \begin{cases}
    n/2          & n \text{ even} \\
    -(n - 1) / 2 & n \text { odd}
  \end{cases}$$

\begin{problem}
Verify that the function $f$ in Example 14.3 is a bijection.
\end{problem}

This example claims that $f(x) = \frac{x}{x + 1}$ is a bijection from $(0, \infty)$ to $(0, 1)$. We can solve for its inverse function to show it is a bijection:
$$y = \frac{x}{x + 1} \implies x = \frac{y}{y - 1}$$



\end{document}