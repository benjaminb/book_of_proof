\documentclass{article}
\usepackage{amsmath} % for advanced math environments
\usepackage{amsfonts} % for math fonts
\usepackage{amssymb} % for math symbols
\usepackage{amsthm} % for theorems and proofs
\usepackage{mathtools} % for mathematical tools
\usepackage{mathrsfs} % for script-like fonts in math
\usepackage{bm} % for bold math symbols
\usepackage{bbm} % for "blackboard-style" characters in math
\usepackage{graphicx} % for including graphics
\usepackage{hyperref} % for including hyperlinks
\usepackage{tcolorbox}
\usepackage{tikz}
\tcbuselibrary{theorems, breakable}
\usepackage{xcolor}
\usepackage[margin=1in]{geometry}

\newcommand{\C}{\mathbb{C}}
\newcommand{\N}{\mathbb{N}}
\newcommand{\Q}{\mathbb{Q}}
\newcommand{\R}{\mathbb{R}}
\newcommand{\Z}{\mathbb{Z}}
\newcommand{\pset}{\mathscr{P}}
\DeclareMathOperator{\lcm}{lcm}

% Define a shortcut for \begin{bmatrix} and \end{bmatrix}
\newcommand{\bmat}[1]{\begin{bmatrix}#1\end{bmatrix}}
\newcommand{\cmat}[1]{\begin{pmatrix}#1\end{pmatrix}}

\newtcolorbox[auto counter]{problem}%
{
    breakable,
    colback=cyan!5,
    colframe=cyan!35!black,
    fonttitle=\bfseries,
    title=Problem~\thetcbcounter,
}

\newtcolorbox{solution}[1]
{
    breakable,
    colback=red!5,
    colframe=red!75!black,
    fonttitle=\bfseries,
    title=Solution: #1,
}

% Title
\title{12.5 Inverse Functions}
\author{Benjamin Basseri}


\begin{document}

\maketitle
\begin{problem}
Check that $f: \Z \to \Z$ defined by $f(n) = 6 - n$ is bijective. Then compute $f^{-1}$.
\end{problem}

To show that it is bijective, show that it is injective and surjective. Injectivity: suppose $f(n) = f(n)$. Then $6 - m = 6 - n$ and $m = n$. Surjectivity: for any $y \in Z$ you want to reach, we solve $y = 6 - n$ for $n$, obtaining $n = 6 -y$. To check, $f(6 - y) = 6 - (6 - y) = y$. Since any $y$ can be reached on $f$, the function is surjective.

To find the inverse, set $y = 6 - n$ and solve for $n$. We just did that, so $f^{-1}(y) = 6 - y$. To check:
$$f^{-1}(f(n)) = f^{-1}(6 - n) = 6 - (6 - n) = n$$
Since $f^{-1} \circ f$ is the identity, this confirms $f^{-1}$ is the correct inverse function.

\begin{problem}
In Exercise 9 of Section 12.2 you proved that $f: \R - \{2\} \to \R - \{5\}$ defined by $f(x) = \frac{5x + 1}{x -2}$ is bijective. Now find its inverse.
\end{problem}


Write $f(x) = y$ and solve for $y$:

\begin{align*}
  \frac{5x + 1}{x - 2} & = y       \\
  5x + 1               & = xy - 2y \\
  2y + 1               & = xy - 5x \\
  2y + 1               & = x(y-5)  \\
  \frac{2y + 1}{y - 5} = x
\end{align*}

Now we can say $f^{-1}(x) = \frac{2x + 1}{x - 5}$. To check:

\begin{align*}
  f(f^{-1}(x)) & = f\left(\frac{2x + 1}{x - 5}\right)                                    \\
               & = \frac{5\left(\frac{2x + 1}{x - 5}\right) + 1}{\frac{2x + 1}{x - 5}-2} \\
               & = \frac{11x}{11}                                                        \\
               & = x
\end{align*}

\begin{problem}
Let $B = \{2^n : n \in \Z\} = \{\ldots, \frac{1}{4}, \frac{1}{2}, 1, 2, 4, 8, \ldots\}$. Show that the function $f: \Z \to B$ defined as $f(n) = 2^n$ is bijective. Then find $f^{-1}$.
\end{problem}

To show bijectivity we could show the inverse, but let's show it's injective and surjective. Injectivity: suppose $f(m) = f(n)$. Then $2^m = 2^n$ and $m$ must equal $n$. The function is surjective by the construction of $B$: the set is all integer powers of 2 and the domain of $f$ is all integers, so any integer power of 2 is in the image of $f$.

The inverse function must map $2^n$ back to $n$, and the $\log_2$ function does this:
$$\log_2 2^n = n \log_2 2 = n\cdot 1 = n$$

\begin{problem}
The function $f: \Z \times\Z \to \Z \times \Z$ defined by the formula $f(m, n) = (5m + 4n, 4m + 3n)$ is bijective. Find its inverse.
\end{problem}

We can solve a system of equations:
\begin{align*}
  5m +  4n & = x \\
  4m +  3n & = y
\end{align*}

Solving the system for $m, n$ gives $m = 4y - 3x, n = 4x - 5y$. Checking we see
\begin{align*}
  f(f^{-1}(m, n)) & = f(4y - 3x, 4x - 5y)                                \\
                  & = (5(4y - 3x) + 4(4x - 5y), 4(4y - 3x) + 3(4x - 5y)) \\
                  & = (20y - 15x + 16x - 20y, 16y - 12x + 12x - 15y)     \\
                  & = (x, y)
\end{align*}
\begin{problem}
Show that the function $f: \R^2 \to \R^2$ defined by the formula $f(x, y) = ((x^2+ 1)y, x^3)$ is bijective. Then find its inverse.
\end{problem}

To show bijectivity, we can show the inverse, but let's show injectivity and surjectivity.

Injectivity: suppose $f(a,b) = f(x, y)$. This means $((a^2 + 1)b, a^3) = ((x^2 + 1)y, x^3)$. But since the coordinates must match this gives $a^3 = x^3$. Since the cubing function is bijective, this means $a = x$. Then the first coordinates being equal means $(a^2 + 1)b = (x^2 + 1)y$. Since $a = x$ we can write this as $(x^2 + 1)b = (x^2 + 1)y$, and by the cancelation law $b = y$.

For surjectivity, the second coordinate can reach any real value by choosing the corresponding cube root for $x$. For the first coordinate, if it is 0 let $y = 0$. Otherwise $(x^2 + 1)y \neq$ which means $x^2 + 1 \neq 0$. Then for any desired value $z$ on the first coordinate, let $y = \frac{z}{x^2 + 1}$. So any value in $\R^2$ can be reached by $f$.

To find the inverse where $f(x, y) = (a, b)$, we must have $x^3 = b$ or $x = b^{1/3}$. To solve for $y$ we have $(x^2 + 1)y = a$, or by substition $(b^{2/3} + 1)y = a$ so that $y = \frac{a}{b^{2/3} + 1}$. Checking we see
\begin{align*}
  f(f^{-1}(a, b)) & = f(b^{1/3}, \frac{a}{b^{2/3} + 1})       \\
                  & = ((b^{2/3} + 1)\frac{a}{b^{2/3} + 1}, b) \\
                  & = (a, b)
\end{align*}

\begin{problem}
Is the function $\theta: \mathcal{P}(\Z) \to \mathcal{P}(\Z)$ defined as $\theta(X) = \overline{X}$ bijective? If so, find its inverse.
\end{problem}

It is bijective because it has an inverse function: itself. The complement of a complement is the original set. Therefore $\theta(\theta(X)) = X$ and $\theta$ composed with $\theta$ is the identity function.

\begin{problem}
Consider the function $f: \R \times \N \to \N \times \R$ defined as $f(x, y) = (y, 3xy)$. Check that this is bijective; find its inverse.
\end{problem}

Injectivity: suppose $f(a, b) = f(x, y)$. Then $(b, 3ab) = (y, 3xy)$. This means $b = y$ and $3ab = 3xb$, so $a = x$. Surjectivity: for any $y \in \N$ needed in the codomain's first coordinate, choose that $y$ as the second input. Then for any $z \in \R$ needed in the codomain's second coordinate, $z$ must equal $3xy$. Then $x = z/3y$ (assuming $0 \not\in \N$). Therefore the function's image is all of $\N \times \R$.

This also give us the inverse function $f^{-1}(x, y) = (y/3x, x)$. Checking:
$$f(f^{-1}(x, y)) = f(y/3x, x) = (x, 3(y/3x)) = (x, y)$$

\begin{problem}
Consider $f: \N \to \Z$ defined as $f(n) = \frac{(-1)^n(2n - 1) + 1}{4}$. Find its inverse.
\end{problem}

Evaluating the first few function values we get:
\begin{align*}
  f(1) & = 0  \\
  f(2) & = 1  \\
  f(3) & = -1 \\
  f(4) & = 2  \\
  f(5) & = -2
\end{align*}

So we see that negative integers come from odd naturals and positive integers come from even ones. From this we can form a piecewise function:
$$f^{-1}(x) = \begin{cases}
    2x,     & x > 0    \\
    1 - 2x, & x \leq 0
  \end{cases}$$
\end{document}